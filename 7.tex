\documentclass[12pt]{article}
\usepackage{multicol,graphicx}
\usepackage[colorlinks,breaklinks,linkcolor=red,citecolor=blue]
{hyperref} 
\def\sectionautorefname~#1\null{\S#1\null}
\usepackage{charter,amsmath,amssymb,breakurl}
\def\equationautorefname~#1\null{(#1)\null}
\def\itemautorefname~#1\null{(#1)\null}
\usepackage{eulervm}
\usepackage[letterpaper,margin=.75in]{geometry}
\renewcommand{\pmod}[1]{\left(\mathsf{mod}\;#1\right)}
\title{Chapter 7}
\renewcommand{\theenumi}{7.\arabic{enumi}}
\author{}\date{}
\let\ord\relax\DeclareMathOperator{\ord}{\mathsf{ord}}
\let\ln\relax\DeclareMathOperator{\ln}{\mathsf{ln}}
\let\deg\relax\DeclareMathOperator{\deg}{\mathsf{deg}}
\let\sin\relax\DeclareMathOperator{\sin}{\mathsf{sin}}
\let\arctan\relax\DeclareMathOperator{\arctan}{\mathsf{arctan}}
\let\cos\relax\DeclareMathOperator{\cos}{\mathsf{cos}}
\let\sec\relax\DeclareMathOperator{\sec}{\mathsf{sec}}
\let\min\relax\DeclareMathOperator*{\min}{\mathsf{min}}
\let\max\relax\DeclareMathOperator*{\max}{\mathsf{max}}
\let\sup\relax\DeclareMathOperator*{\sup}{\mathsf{sup}}
\let\inf\relax\DeclareMathOperator*{\inf}{\mathsf{inf}}
\let\lim\relax\DeclareMathOperator*{\lim}{\mathsf{lim}}
\everymath{\displaystyle}

\makeatletter
\def\legendre@dash#1#2{\hb@xt@#1{%
  \kern-#2\p@
  \cleaders\hbox{\kern.5\p@
    \vrule\@height.2\p@\@depth.2\p@\@width\p@
    \kern.5\p@}\hfil
  \kern-#2\p@
  }}
\def\@legendre#1#2#3#4#5{\mathopen{}\left(
  \sbox\z@{$\genfrac{}{}{0pt}{#1}{#3#4}{#3#5}$}%
  \dimen@=\wd\z@
  \kern-\p@\vcenter{\box0}\kern-\dimen@\vcenter{\legendre@dash\dimen@{#2}}\kern-\p@
  \right)\mathclose{}}
\newcommand\legendre[2]{\mathchoice
  {\@legendre{0}{1}{}{#1}{#2}}
  {\@legendre{1}{.5}{\vphantom{1}}{#1}{#2}}
  {\@legendre{2}{0}{\vphantom{1}}{#1}{#2}}
  {\@legendre{3}{0}{\vphantom{1}}{#1}{#2}}
}
\def\dlegendre{\@legendre{0}{1}{}}
\def\tlegendre{\@legendre{1}{0.5}{\vphantom{1}}}
\makeatother
\begin{document}
\maketitle

\begin{enumerate}
\item %1
Suppose that $F$ is a field and that $H$ is a finite subgroup
of $F^\times$ of size $q$. Then all the elements
of $H$ satisfy $x^q-1$. It follows that
$x^q-1$ has $q$~distinct roots, namely, the elements of $H$.
 If $d$ is a divisor of $q$ then
\[x^q-1=\left(x^d-1\right)\left(x^{q-d}+x^{q-2d}+\cdots+x^d+1\right)\]
so that $x^d-1$ has distinct roots, being a divisor of $x^q-1$.
So there are exactly $d$ elements of $H$ of order dividing $d$.
Let $\psi\left(d\right)$ be the number of elements of order exactly $d$.
Then $d=\sum_{e\mid d}\psi\left(e\right)$ so that by
M\"obius inversion we have
\[\psi\left(d\right)=\sum_{e\mid d}\mu\left(\frac{d}{e}\right)e
=\varphi\left(d\right)\]
for any divisor $d$ of $q$.
In particular $\psi\left(q\right)=\varphi\left(q\right)>1$
unless $q=2$, in which case $H$ is clearly cyclic.

\item %2
\item\label{Number3} %3
Since $q\equiv 1\pmod{n}$ we know that $q=nr+1$ for some $r\in\mathbb{Z}$.
This means that $nr=q-1$ so that $n$ divides $q-1$.
Let $\gamma$ be a generator of the cyclic group $F^\times$.
Write $x=\gamma^y$ and $\alpha=\gamma^m$.
Then $x^n=\alpha$ is solvable if and only if
$\gamma^{ny}=\gamma^m$ is solvable
if and only if $ny\equiv m\pmod{q-1}$ is solvable.
The latter is solvable and has $\left(n,q-1\right)=n$
solutions if and only if $n\mid m$. Otherwise it has no solutions.

\item %4
Let $E=\left\{\alpha\in F^\times
\mid\text{$x^n=\alpha$ for some $x\in F$}\right\}$.
If $\alpha,\beta\in E$ then $\alpha=y^n$ and $\beta=z^n$
for some $y,z\in F$. Then $\alpha\beta^{-1}
=\left(yz^{-1}\right)^n\in E$ shows that $E$ is a subgroup of $F^\times$.
According to \autoref{Number3} $x^n=\alpha$ is solvable
if $m$ is a multiple of $n$, where $m$ is such that $\alpha=\gamma^m$
where $\gamma$ is the generator of $F^\times$.
This means that $m$ can be any of
\[n,2n,3n,\ldots,\frac{q-1}{n}n\]
Thus $\left|E\right|=\frac{q-1}{n}$.

\item %5
Let $\delta$ be a generator of $K^\times$
We can assume that $\gamma=\delta^e$ where
\begin{equation}\label{Epsilon}
e=\frac{\left|K^\times\right|}{\left|F^\times\right|}=\frac{q^n-1}{q-1}
=1+q+\cdots+q^{n-1}
\end{equation}
Note also that $q\equiv 1\pmod{n}$ implies that $q^n\equiv 1\pmod{n}$.
This together with \autoref{Epsilon} means that $e\equiv 0\pmod{n}$.
Thus $e=ns$ for some $s\in\mathbb{Z}$. Together with \autoref{Epsilon}
we have $q^n-1=e\left(q-1\right)=ns\left(q-1\right)$. This shows
that $q^n-1$ is divisible by $n\left(q-1\right)$.

Write $x=\delta^y$ and $\alpha=\gamma^m=\delta^{em}$.
Then $x^n=\alpha$ is solvable if and only if
$\delta^{yn}=\delta^{em}$ is solvable if and only if
$yn\equiv em\pmod{q^n-1}$ is solvable.
The latter has $\left(q^n-1,n\right)=n$ solutions since
$em=\frac{m\left(q^n-1\right)}{q-1}$
is divisible by $n$. Here we use the fact
that $q^n-1$ is divisible by $n\left(q-1\right)$.
\end{enumerate}
\end{document}
