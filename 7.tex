\documentclass[12pt]{article}
\usepackage{multicol,graphicx}
\usepackage[colorlinks,breaklinks,linkcolor=red,citecolor=blue]
{hyperref} 
\def\sectionautorefname~#1\null{\S#1\null}
\usepackage{charter,amsmath,amssymb,breakurl}
\def\equationautorefname~#1\null{(#1)\null}
\def\itemautorefname~#1\null{(#1)\null}
\usepackage{eulervm}
\usepackage[letterpaper,margin=.75in]{geometry}
\renewcommand{\pmod}[1]{\left(\mathsf{mod}\;#1\right)}
\title{Chapter 7}
\renewcommand{\theenumi}{7.\arabic{enumi}}
\author{}\date{}
\let\ord\relax\DeclareMathOperator{\ord}{\mathsf{ord}}
\let\ln\relax\DeclareMathOperator{\ln}{\mathsf{ln}}
\let\deg\relax\DeclareMathOperator{\deg}{\mathsf{deg}}
\let\sin\relax\DeclareMathOperator{\sin}{\mathsf{sin}}
\let\arctan\relax\DeclareMathOperator{\arctan}{\mathsf{arctan}}
\let\cos\relax\DeclareMathOperator{\cos}{\mathsf{cos}}
\let\sec\relax\DeclareMathOperator{\sec}{\mathsf{sec}}
\let\min\relax\DeclareMathOperator*{\min}{\mathsf{min}}
\let\max\relax\DeclareMathOperator*{\max}{\mathsf{max}}
\let\sup\relax\DeclareMathOperator*{\sup}{\mathsf{sup}}
\let\inf\relax\DeclareMathOperator*{\inf}{\mathsf{inf}}
\let\lim\relax\DeclareMathOperator*{\lim}{\mathsf{lim}}
\everymath{\displaystyle}

\makeatletter
\def\legendre@dash#1#2{\hb@xt@#1{%
  \kern-#2\p@
  \cleaders\hbox{\kern.5\p@
    \vrule\@height.2\p@\@depth.2\p@\@width\p@
    \kern.5\p@}\hfil
  \kern-#2\p@
  }}
\def\@legendre#1#2#3#4#5{\mathopen{}\left(
  \sbox\z@{$\genfrac{}{}{0pt}{#1}{#3#4}{#3#5}$}%
  \dimen@=\wd\z@
  \kern-\p@\vcenter{\box0}\kern-\dimen@\vcenter{\legendre@dash\dimen@{#2}}\kern-\p@
  \right)\mathclose{}}
\newcommand\legendre[2]{\mathchoice
  {\@legendre{0}{1}{}{#1}{#2}}
  {\@legendre{1}{.5}{\vphantom{1}}{#1}{#2}}
  {\@legendre{2}{0}{\vphantom{1}}{#1}{#2}}
  {\@legendre{3}{0}{\vphantom{1}}{#1}{#2}}
}
\def\dlegendre{\@legendre{0}{1}{}}
\def\tlegendre{\@legendre{1}{0.5}{\vphantom{1}}}
\makeatother
\begin{document}
\maketitle

\begin{enumerate}
\item %1
Suppose that $F$ is a field and that $H$ is a finite subgroup
of $F^\times$ of size $q$. Then all the elements
of $H$ satisfy $x^q-1$. It follows that
$x^q-1$ has $q$~distinct roots, namely, the elements of $H$.
 If $d$ is a divisor of $q$ then
\[x^q-1=\left(x^d-1\right)\left(x^{q-d}+x^{q-2d}+\cdots+x^d+1\right)\]
so that $x^d-1$ has distinct roots, being a divisor of $x^q-1$.
So there are exactly $d$ elements of $H$ of order dividing $d$.
Let $\psi\left(d\right)$ be the number of elements of order exactly $d$.
Then $d=\sum_{e\mid d}\psi\left(e\right)$ so that by
M\"obius inversion we have
\[\psi\left(d\right)=\sum_{e\mid d}\mu\left(\frac{d}{e}\right)e
=\varphi\left(d\right)\]
for any divisor $d$ of $q$.
In particular $\psi\left(q\right)=\varphi\left(q\right)>1$
unless $q=2$, in which case $H$ is clearly cyclic.
\end{enumerate}
\end{document}
