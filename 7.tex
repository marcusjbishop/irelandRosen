\documentclass[12pt]{article}
\usepackage{multicol,graphicx}
\usepackage[colorlinks,breaklinks,linkcolor=red,citecolor=blue]
{hyperref} 
\def\sectionautorefname~#1\null{\S#1\null}
\usepackage{charter,amsmath,amssymb,breakurl}
\def\equationautorefname~#1\null{(#1)\null}
\def\itemautorefname~#1\null{(#1)\null}
\usepackage{eulervm}
\usepackage[letterpaper,margin=.75in]{geometry}
\renewcommand{\pmod}[1]{\left(\mathsf{mod}\;#1\right)}
\title{Chapter 7}
\renewcommand{\theenumi}{7.\arabic{enumi}}
\author{}\date{}
\let\ord\relax\DeclareMathOperator{\ord}{\mathsf{ord}}
\let\ln\relax\DeclareMathOperator{\ln}{\mathsf{ln}}
\let\deg\relax\DeclareMathOperator{\deg}{\mathsf{deg}}
\let\sin\relax\DeclareMathOperator{\sin}{\mathsf{sin}}
\let\arctan\relax\DeclareMathOperator{\arctan}{\mathsf{arctan}}
\let\cos\relax\DeclareMathOperator{\cos}{\mathsf{cos}}
\let\sec\relax\DeclareMathOperator{\sec}{\mathsf{sec}}
\let\min\relax\DeclareMathOperator*{\min}{\mathsf{min}}
\let\max\relax\DeclareMathOperator*{\max}{\mathsf{max}}
\let\sup\relax\DeclareMathOperator*{\sup}{\mathsf{sup}}
\let\inf\relax\DeclareMathOperator*{\inf}{\mathsf{inf}}
\let\lim\relax\DeclareMathOperator*{\lim}{\mathsf{lim}}
\everymath{\displaystyle}

\makeatletter
\def\legendre@dash#1#2{\hb@xt@#1{%
  \kern-#2\p@
  \cleaders\hbox{\kern.5\p@
    \vrule\@height.2\p@\@depth.2\p@\@width\p@
    \kern.5\p@}\hfil
  \kern-#2\p@
  }}
\def\@legendre#1#2#3#4#5{\mathopen{}\left(
  \sbox\z@{$\genfrac{}{}{0pt}{#1}{#3#4}{#3#5}$}%
  \dimen@=\wd\z@
  \kern-\p@\vcenter{\box0}\kern-\dimen@\vcenter{\legendre@dash\dimen@{#2}}\kern-\p@
  \right)\mathclose{}}
\newcommand\legendre[2]{\mathchoice
  {\@legendre{0}{1}{}{#1}{#2}}
  {\@legendre{1}{.5}{\vphantom{1}}{#1}{#2}}
  {\@legendre{2}{0}{\vphantom{1}}{#1}{#2}}
  {\@legendre{3}{0}{\vphantom{1}}{#1}{#2}}
}
\def\dlegendre{\@legendre{0}{1}{}}
\def\tlegendre{\@legendre{1}{0.5}{\vphantom{1}}}
\makeatother
\begin{document}
\maketitle

\begin{enumerate}
\item %1
Suppose that $F$ is a field and that $H$ is a finite subgroup
of $F^\times$ of size $q$. Then all the elements
of $H$ satisfy $x^q-1$. It follows that
$x^q-1$ has $q$~distinct roots, namely, the elements of $H$.
 If $d$ is a divisor of $q$ then
\[x^q-1=\left(x^d-1\right)\left(x^{q-d}+x^{q-2d}+\cdots+x^d+1\right)\]
so that $x^d-1$ has distinct roots, being a divisor of $x^q-1$.
So there are exactly $d$ elements of $H$ of order dividing $d$.
Let $\psi\left(d\right)$ be the number of elements of order exactly $d$.
Then $d=\sum_{e\mid d}\psi\left(e\right)$ so that by
M\"obius inversion we have
\[\psi\left(d\right)=\sum_{e\mid d}\mu\left(\frac{d}{e}\right)e
=\varphi\left(d\right)\]
for any divisor $d$ of $q$.
In particular $\psi\left(q\right)=\varphi\left(q\right)>1$
unless $q=2$, in which case $H$ is clearly cyclic.

\item %2
\item\label{Number3} %3
Since $q\equiv 1\pmod{n}$ we know that $q=nr+1$ for some $r\in\mathbb{Z}$.
This means that $nr=q-1$ so that $n$ divides $q-1$.
Let $\gamma$ be a generator of the cyclic group $F^\times$.
Write $x=\gamma^y$ and $\alpha=\gamma^m$.
Then $x^n=\alpha$ is solvable if and only if
$\gamma^{ny}=\gamma^m$ is solvable
if and only if $ny\equiv m\pmod{q-1}$ is solvable.
The latter is solvable and has $\left(n,q-1\right)=n$
solutions if and only if $n\mid m$. Otherwise it has no solutions.

\item %4
Let $E=\left\{\alpha\in F^\times
\mid\text{$x^n=\alpha$ for some $x\in F$}\right\}$.
If $\alpha,\beta\in E$ then $\alpha=y^n$ and $\beta=z^n$
for some $y,z\in F$. Then $\alpha\beta^{-1}
=\left(yz^{-1}\right)^n\in E$ shows that $E$ is a subgroup of $F^\times$.
According to \autoref{Number3} $x^n=\alpha$ is solvable
if $m$ is a multiple of $n$, where $m$ is such that $\alpha=\gamma^m$
where $\gamma$ is the generator of $F^\times$.
This means that $m$ can be any of
\[n,2n,3n,\ldots,\frac{q-1}{n}n\]
Thus $\left|E\right|=\frac{q-1}{n}$.

\item %5
Let $\delta$ be a generator of $K^\times$
We can assume that $\gamma=\delta^e$ where
\begin{equation}\label{Epsilon}
e=\frac{\left|K^\times\right|}{\left|F^\times\right|}=\frac{q^n-1}{q-1}
=1+q+\cdots+q^{n-1}
\end{equation}
Note also that $q\equiv 1\pmod{n}$ implies that $q^n\equiv 1\pmod{n}$.
This together with \autoref{Epsilon} means that $e\equiv 0\pmod{n}$.
Thus $e=ns$ for some $s\in\mathbb{Z}$. Together with \autoref{Epsilon}
we have $q^n-1=e\left(q-1\right)=ns\left(q-1\right)$. This shows
that $q^n-1$ is divisible by $n\left(q-1\right)$.

Write $x=\delta^y$ and $\alpha=\gamma^m=\delta^{em}$.
Then $x^n=\alpha$ is solvable if and only if
$\delta^{yn}=\delta^{em}$ is solvable if and only if
$yn\equiv em\pmod{q^n-1}$ is solvable.
The latter has $\left(q^n-1,n\right)=n$ solutions since
$em=\frac{m\left(q^n-1\right)}{q-1}$
is divisible by $n$. Here we use the fact
that $q^n-1$ is divisible by $n\left(q-1\right)$.

\item\label{NotSquare} %6
Let $q=\left|F\right|$ and let $\gamma$ be a generator
of $F^\times$.  Write $x=\gamma^y$ and let
$m\in\mathbb{Z}$ be such that $\alpha=\gamma^m$.
That $x^2=\alpha$ has no solution $x\in F$ implies that
$\gamma^{2y}=\gamma^{m}$ has no solution $y\in\mathbb{Z}$.
This in turn implies that $2y\equiv m\pmod{q-1}$ has no solution $y$.
This means that $d=\left(q-1,2\right)$ does not divide $m$.
If $q$ were even, then $d$ would be $1$, which {\em does} divide $m$.
This means that $q$ is odd and $d=2$.
We remark that we have recovered the fact that all the
elements of a field of characteristic~$2$ are squares,
a special case of the fact that all the elements
of a field of characteristic~$p$ are $p$th powers.
Since $d=2$ does not divide $m$, we also know that $m$ is odd.

Now let $\delta$ be a generator of $K$
and assume that $\gamma$ was chosen above such that $\gamma=\delta^e$
where $e=\frac{q^3-1}{q-1}=q^2+q+1$.
Then $x^2=\alpha$ has a solution $x\in K$
if and only if $\delta^{2y}=\delta^{em}$ has a solution $y\in\mathbb{Z}$
if and only if $2y\equiv 3m\pmod{q^3-1}$ has a solution $y\in\mathbb{Z}$.
The latter holds if and only if $2=\left(q^3-1\right)$ divides $em$.
This is not the case since $em$ is odd. We conclude that
Then $x^2=\alpha$ has no solution $x\in K$.

\item We repeat the argument in \autoref{NotSquare}
except that $K$ is an extension of degree~$n$ rather than~$3$.
This means that
$e=\frac{q^3-1}{q-1}=q^{n-1}+q^{n-2}+\cdots+q+1$ has
$n$~terms, all odd.
So $e$ is even if and only if $n$ is even.
Thus $em$ is even if and only
if $n$ is even. Thus $2y\equiv nm\pmod{q^n-1}$ has a solution
if and only if $n$ is even.

\item %8
As noted in \autoref{NotSquare} all the elements of
a field of characteristic~2 are squares.

\item %9
Let $\gamma$ be a generator of $F^\times$
and $\delta$ a generator of $K$.
Assume also that $\gamma=\delta^e$
where $e=\frac{q^d-1}{q-1}$ where $d=\left|K:F\right|$.
Let $m\in\mathbb{Z}$ be such that $\alpha=\gamma^m$.
That $\alpha=x^n$ has no solution in $F$ implies that
$\gamma^m=\gamma^{ny}$ has no solution $y\in\mathbb{Z}$.
In turn $m\equiv ny\pmod{q-1}$ has no solution $y\in\mathbb{Z}$
implies that $\left(q-1,n\right)=n$ does not divide $m$.

Now $\alpha=x^n$ has a solution in $K$ if and only if
$\delta^{em}=\delta^{ny}$ has a solution $y\in\mathbb{Z}$
if and only if $em\equiv ny\pmod{q^d-1}$
has a solution $y\in\mathbb{Z}$
if and only if $\left(q^d-1,n\right)=n$ divides $em$.
This holds if and only if $em\equiv 0\pmod{n}$.
Now $q\equiv 1\pmod{n}$ implies that
\[e\equiv\left(q^{d-1}+q^{d-2}+\cdots+q+1\right)\equiv d\pmod{n}.\]
But $d\not\equiv 0$ by the assumption
that $\left(d,n\right)=1$ and $m\not\equiv 0$ by the observation
above that $n$ does not divide $m$.
We conclude that $em\not\equiv 0$ so that $\alpha=x^n$
has no solution in $K$.

\item %10
Let $\gamma$ be a generator of $F^\times$
and $\delta$ a generator of $K$.
Assume also that $\gamma=\delta^e$
where $e=\frac{q^2-1}{q-1}=q+1$.
Let $\beta\in K$.
If $\beta=0$ then naturally $\beta^{q+1}=0\in F$.
If $\beta=\delta^m$ is any element of $K^\times$
then $\beta^{q+1}=\delta^{\left(q+1\right)m}=\gamma^m\in F$.
On the other hand, suppose $\alpha$ is any element of $F$.
If $\alpha=0$ then $0^{q+1}=\alpha$.
If $\alpha=\gamma^m\in F^\times$
then $\beta^{q+1}=\alpha$ where $\beta=\delta^m$.

\item %11
\item %12
\item %13
\item %14
We first prove that
\begin{equation}\label{Theorem2}
x^{q^n}-x=\prod_{d\mid n}G_d\left(x\right)
\end{equation}
where $G_d\left(x\right)$ is the product
of the monic irreducible polynomials of degree~$d$ over $\mathbb{F}_q$.
First observe that $x^{q^n}-x$ is separable, being
relatively prime to its derivative $-1$. This means that if
$f\left(x\right)$ is any irreducible factor of $x^{q^n}-x$
then $\left(f\left(x\right)\right)^2$ is not a factor.

Suppose that $f\left(x\right)$ is a monic irreducible polynomial
of degree~$d$ over $\mathbb{F}_q$.
Let $\alpha$ be a root of $f\left(x\right)$ and put
$K=\mathbb{F}_q\left(\alpha\right)$. Then $\left|K:\mathbb{F}_q\right|=d$
so that all the elements of $K$ satisfy $x^{q^d}-x$.
We will show that $f\left(x\right)$ divides $x^{q^n}-x$
if and only if $d\mid n$.

Suppose that $f\left(x\right)$ divides $x^{q^n}-x$.
Then $x^{q^n}-x=f\left(x\right)g\left(x\right)$.
Substituting $x=\alpha$ shows that $\alpha^{q^n}=\alpha$.
Now if $\beta=\sum_{i=0}^{d-1}b_i\alpha^i$ is any element
of~$K$ with $b_0,b_1,\ldots,b_{d-1}\in\mathbb{F}_q$ then
\[\beta^{q^n}=\left(\sum_{i=0}^{d-1}b_i\alpha^i\right)^{q^n}
=\sum_{i=0}^{d-1}b_i\left(\alpha^{q^n}\right)^i
=\sum_{i=0}^{d-1}b_i\alpha=\beta\]
shows that all the elements of $K$ satisfy $x^{q^n}-x$.
This means that $x^{q^d}-x$ divides $x^{q^n}-x$ so that $d\mid n$.

Conversely, suppose that $d\mid n$. This means that $x^{q^d}-x$
divides $x^{q^n}-x$. Now since $\alpha$ satisfies $x^{q^d}-x$
and $f\left(x\right)$ is the minimal polynomial of $\alpha$
we know that $f\left(x\right)$ divides $x^{q^d}-x$ which in
turn divides $x^{q^d}-x$. This shows that $f\left(x\right)$
divides $x^{q^n}-x$ and proves \autoref{Theorem2}.

Taking degrees of both sides of \autoref{Theorem2} gives
\[p^n=\sum_{d\mid n}dN_d\]
where $N_d$ is the number of monic irreducible polynomials
of degree~$d$ over~$\mathbb{F}_q$.
By M\"obius inversion we have
\begin{equation}\label{Mobius}
nN_n=\sum_{d\mid n}\mu\left(\frac{n}{d}\right)p^d.
\end{equation}
We conclude that $N_n\ne 0$ for any $n\ge 1$ since
the right hand side of \autoref{Mobius} is a sum
of powers of $p$ with coefficients $\pm 1$.
Thus, there exist monic irreducible polynomials of degree~$n$
over~$\mathbb{F}_q$ for any $n\ge 1$.

\item %14
Suppose that $K$ is an extension of $\mathbb{F}_q$ of degree $f$. Then
$q^f\equiv 1\pmod{n}$ implies that $n$ divides $q^f-1$
which in turn implies that $x^n-1$ divides $x^{q^f-1}-1$.
Then since all the nonzero elements of $K$ satisfy $x^{q^n-1}-1$
we see that $x^n-1$ splits completely in $K$.

On the other hand, if $g\in\mathbb{N}$ is such that
$x^n-1$ splits completely in an extension of $\mathbb{F}_q$
of degree~$g$ then $x^n-1$ divides $x^{q^g-1}-1$
which implies that $n$ divides $q^g-1$ which in turn implies
that $q^g\equiv 1\pmod{n}$. Thus $g\ge f$.

\item %16
Multiplying all the polynomials of degree~$3$ by all the polynomials
of degree~$1$ and multiplying all the polynomials of degree~$2$
by all the polynomials of degree~$2$
produces all the reducible polynomials of degree~$4$.
The polynomials of degree~$4$ not listed above are
$x^4+x+1$, $x^4+x^3+1$, and $x^4+x^3+x^2+x+1$.

\item %17
By \autoref{Mobius}
$qN_q=\sum_{d\mid q}\mu\left(\frac{q}{d}\right)p^d
=p^d-p$ so that $N_q=\frac{p^d-p}{q}$.

\item %18
\item %19
$\sum_f\left|f\right|^{-s}
=\sum_{n\ge 0}\sum_{\deg{f}=n}\left|f\right|^{-s}
=\sum_{n\ge 0}\sum_{\deg{f}=n}q^{-sn}
=\sum_{n\ge 0}q^{n}q^{-sn}
=\sum_{n\ge 0}q^{n\left(1-s\right)}
=\frac{1}{1-q^{1-s}}$

\item %20
$\sum_fd\left(f\right)\left|f\right|^{-s}$
counts $q^{-s\deg{f}}$ for each divisor of each monic polynomial $f$.
Another way to calculate the sum is to count
$q^{-s\left(\deg{f}+\deg{g}\right)}$ for every monic polynomial $g$
for every monic polynomial $f$. This gives
\begin{align*}
\sum_{m\ge 0}\sum_{n\ge 0}q^{m+n-s\left(m+n\right)}
&=\sum_{k\ge 0}\left(k+1\right)q^{k-sk}\\
&=\frac{1}{1-s}\frac{d}{dq}\sum_{k\ge 1}
q^{\left(k+1\right)\left(1-s\right)}\\
&=\frac{1}{1-s}\frac{d}{dq}\frac{q^{2\left(1-s\right)}}{1-q^{1-s}}
\end{align*}

\item %21
Let $\alpha\in\mathbb{F}_{p^n}$
and suppose that $g_\alpha\left(x\right)
=\sum_{i=0}^nb_ix^i\in\mathbb{F}_p\left[x\right]$
is the minimal polynomial of $\alpha$ over $\mathbb{F}_p$.
Then $0=g_\alpha\left(\alpha\right)=\sum_{i=0}^nb_i\alpha^i$.
Now for any $0\le j\le n-1$ we have
\[g_\alpha\left(\alpha^{p^j}\right)
=\sum_{i=0}^nb_i\alpha^{ip^j}
=\left(\sum_{i=0}^nb_i\alpha^i\right)^{p^j}=0.\]
This shows that $\alpha,\alpha^p,\alpha^{p^2},\ldots,\alpha^{p^{n-1}}$
are the roots of $g_\alpha\left(x\right)$ so that
\[f\left(x\right)=\prod_{j=0}^{n-1}\left(x-\alpha^{p^j}\right)\]
coincides with $g_\alpha\left(x\right)$, both
polynomials being monic and having the same roots.
We conclude that
$f\left(x\right)=g_\alpha\left(x\right)
\in\mathbb{F}_p\left[x\right]$.

\item %22
\begin{enumerate}
\item $\mathsf{tr}\left(\alpha+\beta\right)
=\sum_{j=0}^{n-1}\left(\alpha+\beta\right)^{p^j}
=\sum_{j=0}^{n-1}\left(\alpha^{p^j}+\beta^{p^j}\right)
=\mathsf{tr}\left(\alpha\right)+\mathsf{tr}\left(\beta\right)$
\item $\mathsf{tr}\left(a\alpha\right)
=\sum_{j=0}^{n-1}\left(a\alpha\right)^{p^j}
=\sum_{j=0}^{n-1}a\alpha^{p^j}=a\mathsf{tr}\left(\alpha\right)$.
\end{enumerate}

\end{enumerate}
\end{document}
