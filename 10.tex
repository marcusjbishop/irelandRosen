\documentclass[12pt]{article}
\usepackage{multicol,graphicx}
\usepackage[colorlinks,breaklinks,linkcolor=red,citecolor=blue]
{hyperref} 
\def\sectionautorefname~#1\null{\S#1\null}
\usepackage{charter,amsmath,amssymb,breakurl}
\def\equationautorefname~#1\null{(#1)\null}
\def\itemautorefname~#1\null{(#1)\null}
\usepackage{eulervm}
\usepackage[letterpaper,margin=.75in]{geometry}
\renewcommand{\pmod}[1]{\left(\mathsf{mod}\;#1\right)}
\title{Chapter 10}
\renewcommand{\theenumi}{10.\arabic{enumi}}
\author{}\date{}
\let\ker\relax\DeclareMathOperator{\ker}{\mathsf{ker}}
\let\ord\relax\DeclareMathOperator{\ord}{\mathsf{ord}}
\let\ln\relax\DeclareMathOperator{\ln}{\mathsf{ln}}
\let\deg\relax\DeclareMathOperator{\deg}{\mathsf{deg}}
\let\sin\relax\DeclareMathOperator{\sin}{\mathsf{sin}}
\let\arctan\relax\DeclareMathOperator{\arctan}{\mathsf{arctan}}
\let\cos\relax\DeclareMathOperator{\cos}{\mathsf{cos}}
\let\sec\relax\DeclareMathOperator{\sec}{\mathsf{sec}}
\let\min\relax\DeclareMathOperator*{\min}{\mathsf{min}}
\let\max\relax\DeclareMathOperator*{\max}{\mathsf{max}}
\let\sup\relax\DeclareMathOperator*{\sup}{\mathsf{sup}}
\let\inf\relax\DeclareMathOperator*{\inf}{\mathsf{inf}}
\let\lim\relax\DeclareMathOperator*{\lim}{\mathsf{lim}}
\everymath{\displaystyle}

\makeatletter
\def\legendre@dash#1#2{\hb@xt@#1{%
  \kern-#2\p@
  \cleaders\hbox{\kern.5\p@
    \vrule\@height.2\p@\@depth.2\p@\@width\p@
    \kern.5\p@}\hfil
  \kern-#2\p@
  }}
\def\@legendre#1#2#3#4#5{\mathopen{}\left(
  \sbox\z@{$\genfrac{}{}{0pt}{#1}{#3#4}{#3#5}$}%
  \dimen@=\wd\z@
  \kern-\p@\vcenter{\box0}\kern-\dimen@\vcenter{\legendre@dash\dimen@{#2}}\kern-\p@
  \right)\mathclose{}}
\newcommand\legendre[2]{\mathchoice
  {\@legendre{0}{1}{}{#1}{#2}}
  {\@legendre{1}{.5}{\vphantom{1}}{#1}{#2}}
  {\@legendre{2}{0}{\vphantom{1}}{#1}{#2}}
  {\@legendre{3}{0}{\vphantom{1}}{#1}{#2}}
}
\def\dlegendre{\@legendre{0}{1}{}}
\def\tlegendre{\@legendre{1}{0.5}{\vphantom{1}}}
\makeatother
\begin{document}
\maketitle

\begin{enumerate}
\item %1

\item %2
\item %3
We will prove that $\left|\mathbb{P}^{n}\right|
=1+q+\cdots+q^n$ by induction.
If $n=0$ then the formula holds, since $\mathbb{P}^0$ has exactly
one point.
We assume that $\left|\mathbb{P}^{n-1}\right|
=1+q+\cdots+q^{n-1}$ by induction.
Then since $\mathbb{P}^n=\mathbb{P}^{n-1}\coprod\mathbb{A}^n$
we have $\left|\mathbb{P}^n\right|
=\left|\mathbb{P}^{n-1}\right|+\left|\mathbb{A}^n\right|
=1+q+\cdots+q^n$.

\item %4
Write $f\left(x_0,x_1,\ldots,x_n\right)
=a_0x_0+a_1x_1+\cdots+a_nx_n$.
Although not mentioned, we assume that $f\ne 0$
since otherwise $\overline{H}_f=\mathbb{P}^n$,
which does {\em not} have the same number of points
as $\mathbb{P}^{n-1}$. We therefore assume that one of
$a_0,a_1,\ldots,a_n$ is nonzero, say $a_n$. Define
$\varphi:\mathbb{P}^{n-1}\to\overline{H}_f$ by
\[\varphi\left[x_0,x_1,\ldots,x_{n-1}\right]
=\left[x_0,x_1,\ldots,x_{n-1},-\frac{a_0x_0+\cdots+a_{n-1}x_{n-1}}{a_n}
\right].\]
Then $\varphi$ is well-defined since
\[\varphi\left[\alpha x_0,\ldots,\alpha x_{n-1}\right]
=\left[\alpha x_0,\ldots,\alpha x_{n-1},
-\frac{a_0\alpha x_0+\cdots+a_{n-1}\alpha x_{n-1}}{a_n}\right]
=\varphi\left[x_0,x_1,\ldots,x_{n-1}\right]\]
for any $\alpha\ne 0$.
Now if $\varphi\left(x\right)=\varphi\left(y\right)$
then there exists $\beta\ne 0$ such that
\[\left[x_0,\ldots,x_{n-1},-\frac{a_0x_0
+\cdots+a_{n-1}x_{n-1}}{a_n}\right]
=\left[\beta y_0,\ldots,\beta y_{n-1},-\beta\frac{a_0y_0
+\cdots+a_{n-1}y_{n-1}}{a_n}\right]\]
In particular this means that $x_i=\beta y_i$
for all $0\le i\le n-1$ so that $x=y$.
This shows that $\varphi$ is injective.
Finally, suppose that $y$ is any point on $\overline{H}_f$.
This means that $a_0y_0+\cdots+a_ny_n=0$
so that $y_n=-\frac{a_0y_0+\cdots+a_{n-1}y_{n-1}}{a_n}$.
Then putting $x=\left[y_0,\ldots,y_{n-1}\right]$ we have
\[\varphi\left(x\right)
=\left[y_0,\ldots,y_{n-1},-\frac{a_0y_0+\cdots+a_{n-1}y_{n-1}}{a_n}
\right]=y.\]
This shows that $\varphi$ is surjective.
\end{enumerate}
\end{document}
