\documentclass[12pt]{article}
\usepackage{multicol,graphicx}
\usepackage[colorlinks,breaklinks,linkcolor=red,citecolor=blue]
{hyperref} 
\def\sectionautorefname~#1\null{\S#1\null}
\usepackage{charter,amsmath,amssymb,breakurl}
\def\equationautorefname~#1\null{(#1)\null}
\def\itemautorefname~#1\null{(#1)\null}
\usepackage{eulervm}
\usepackage[letterpaper,margin=.75in]{geometry}
\renewcommand{\pmod}[1]{\left(\mathsf{mod}\;#1\right)}
\title{Chapter 10}
\renewcommand{\theenumi}{10.\arabic{enumi}}
\author{}\date{}
\let\det\relax\DeclareMathOperator{\det}{\mathsf{det}}
\let\ker\relax\DeclareMathOperator{\ker}{\mathsf{ker}}
\let\ord\relax\DeclareMathOperator{\ord}{\mathsf{ord}}
\let\ln\relax\DeclareMathOperator{\ln}{\mathsf{ln}}
\let\deg\relax\DeclareMathOperator{\deg}{\mathsf{deg}}
\let\sin\relax\DeclareMathOperator{\sin}{\mathsf{sin}}
\let\arctan\relax\DeclareMathOperator{\arctan}{\mathsf{arctan}}
\let\cos\relax\DeclareMathOperator{\cos}{\mathsf{cos}}
\let\sec\relax\DeclareMathOperator{\sec}{\mathsf{sec}}
\let\min\relax\DeclareMathOperator*{\min}{\mathsf{min}}
\let\max\relax\DeclareMathOperator*{\max}{\mathsf{max}}
\let\sup\relax\DeclareMathOperator*{\sup}{\mathsf{sup}}
\let\inf\relax\DeclareMathOperator*{\inf}{\mathsf{inf}}
\let\lim\relax\DeclareMathOperator*{\lim}{\mathsf{lim}}
\everymath{\displaystyle}

\makeatletter
\def\legendre@dash#1#2{\hb@xt@#1{%
  \kern-#2\p@
  \cleaders\hbox{\kern.5\p@
    \vrule\@height.2\p@\@depth.2\p@\@width\p@
    \kern.5\p@}\hfil
  \kern-#2\p@
  }}
\def\@legendre#1#2#3#4#5{\mathopen{}\left(
  \sbox\z@{$\genfrac{}{}{0pt}{#1}{#3#4}{#3#5}$}%
  \dimen@=\wd\z@
  \kern-\p@\vcenter{\box0}\kern-\dimen@\vcenter{\legendre@dash\dimen@{#2}}\kern-\p@
  \right)\mathclose{}}
\newcommand\legendre[2]{\mathchoice
  {\@legendre{0}{1}{}{#1}{#2}}
  {\@legendre{1}{.5}{\vphantom{1}}{#1}{#2}}
  {\@legendre{2}{0}{\vphantom{1}}{#1}{#2}}
  {\@legendre{3}{0}{\vphantom{1}}{#1}{#2}}
}
\def\dlegendre{\@legendre{0}{1}{}}
\def\tlegendre{\@legendre{1}{0.5}{\vphantom{1}}}
\makeatother
\begin{document}
\maketitle

\begin{enumerate}
\item %1
Let $K$ be an infinite field and $f$ a non-zero polynomial
of degree~$m$ in $K\left[x_1,x_2,\ldots,x_n\right]$.
We will show that $f$ is not identically
zero on $\mathbb{A}^n$. The proof if by induction on~$n$.

Suppose that $n=1$. This means that $f$ is a univariate polynomial
of degree~$m$ over~$K$. Recall that a non-zero univariate polynomial
of degree~$m$ over {\em any} field can have at most $m$~roots
in that field. Now if $f$ were identically zero on $\mathbb{A}^1$
then $f$~would have more than $m$~roots. This would mean that
$f$ is the zero polynomial, a contradiction.

Now suppose that the claim holds when $f$ is a polynomial
in fewer than $n$~variables. Then we can write
\[f\left(x_1,x_2,\ldots,x_n\right)=\sum_{i=0}^m
g_i\left(x_1,x_2,\ldots,x_{n-1}\right)x_n^i\]
for some polynomials $g_i\in K\left[x_1,\ldots,x_{n-1}\right]$.
Now if $f$~ were identically zero on $\mathbb{A}^n$ then
for any $a_1,a_2,\ldots a_{n-1}\in K$
\[\sum_{i=0}^m g_i\left(a_1,a_2,\ldots,a_{n-1}\right)x_n^i\]
would be a univariate polynomial of degree~$m$ in $x_n$
which vanishes for all values of $x_n$. This would mean
that $g_i\left(a_1,a_2,\ldots,a_{n-1}\right)=0$ for all~$i$.
But since $a_1,a_2,\ldots,a_{n-1}$ were arbitrary, this would
mean by induction that each $g_i$ is the zero polynomial.
In turn, this would mean that~$f$
is the zero polynomial, a contradiction.

\item %2
\item %3
We will prove that $\left|\mathbb{P}^{n}\right|
=1+q+\cdots+q^n$ by induction.
If $n=0$ then the formula holds, since $\mathbb{P}^0$ has exactly
one point.
We assume that $\left|\mathbb{P}^{n-1}\right|
=1+q+\cdots+q^{n-1}$ by induction.
Then since $\mathbb{P}^n=\mathbb{P}^{n-1}\coprod\mathbb{A}^n$
we have $\left|\mathbb{P}^n\right|
=\left|\mathbb{P}^{n-1}\right|+\left|\mathbb{A}^n\right|
=1+q+\cdots+q^n$.

\item %4
Write $f\left(x_0,x_1,\ldots,x_n\right)
=a_0x_0+a_1x_1+\cdots+a_nx_n$.
Although not mentioned, we assume that $f\ne 0$
since otherwise $\overline{H}_f=\mathbb{P}^n$,
which does {\em not} have the same number of points
as $\mathbb{P}^{n-1}$. We therefore assume that one of
$a_0,a_1,\ldots,a_n$ is nonzero, say $a_n$. Define
$\varphi:\mathbb{P}^{n-1}\to\overline{H}_f$ by
\[\varphi\left[x_0,x_1,\ldots,x_{n-1}\right]
=\left[x_0,x_1,\ldots,x_{n-1},-\frac{a_0x_0+\cdots+a_{n-1}x_{n-1}}{a_n}
\right].\]
Then $\varphi$ is well-defined since
\[\varphi\left[\alpha x_0,\ldots,\alpha x_{n-1}\right]
=\left[\alpha x_0,\ldots,\alpha x_{n-1},
-\frac{a_0\alpha x_0+\cdots+a_{n-1}\alpha x_{n-1}}{a_n}\right]
=\varphi\left[x_0,x_1,\ldots,x_{n-1}\right]\]
for any $\alpha\ne 0$.
Now if $\varphi\left(x\right)=\varphi\left(y\right)$
then there exists $\beta\ne 0$ such that
\[\left[x_0,\ldots,x_{n-1},-\frac{a_0x_0
+\cdots+a_{n-1}x_{n-1}}{a_n}\right]
=\left[\beta y_0,\ldots,\beta y_{n-1},-\beta\frac{a_0y_0
+\cdots+a_{n-1}y_{n-1}}{a_n}\right]\]
In particular this means that $x_i=\beta y_i$
for all $0\le i\le n-1$ so that $x=y$.
This shows that $\varphi$ is injective.
Finally, suppose that $y$ is any point on $\overline{H}_f$.
This means that $a_0y_0+\cdots+a_ny_n=0$
so that $y_n=-\frac{a_0y_0+\cdots+a_{n-1}y_{n-1}}{a_n}$.
Then putting $x=\left[y_0,\ldots,y_{n-1}\right]$ we have
\[\varphi\left(x\right)
=\left[y_0,\ldots,y_{n-1},-\frac{a_0y_0+\cdots+a_{n-1}y_{n-1}}{a_n}
\right]=y.\]
This shows that $\varphi$ is surjective.

\item %5
Let $g=a_0x_0+a_1x_1+a_2x_2$. We can assume that
$g\ne 0$ since otherwise $\overline{H}_{\overline{g}}$
would be all of $\mathbb{P}^2$,
in which case the assertion is false.
We therefore assume that one of $a_0,a_1,a_2$ is nonzero,
say $a_1\ne 0$. The points at infinity satisfying
both equations are those of the form $\left[0,x,y\right]$.
Then since $g\left(0,x,y\right)=0$ we have
$x_1=-\frac{a_2x_2}{a_1}$. This means that the two curves
have a single common point at infinity
$\left[0,-\frac{a_2x_2}{a_1},x_2\right]$ if
$f\left(0,-\frac{a_2x_2}{a_1},x_2\right)=0$
or no points at infinity otherwise.
\\
Now let $\left[1,x_1,x_2\right]$ be a common affine point
on the two curves. Then since $g\left(1,x_1,x_2\right)=0$
we have $x_1=\frac{a_0-a_2x_2}{a_1}$. Now since
$f\left(1,\frac{a_0-a_2x_2}{a_1},x_2\right)$ is a univariate
polynomial of degree $m$ in $x_2$, it has at most
$n$~solutions.

We therefore have too many points\dots

\item %6
The elements of $\mathsf{GL}_n\left(F\right)$
correspond with ordered bases $\left(v_1,v_2,\ldots,v_n\right)$ of $F^n$.
To count these, we first observe that $v_1$ can be any of the
$q^n-1$ nonzero vectors. Then $v_2$ can be any vector which
is not a multiple of $v_1$. There are $q^n-q$ such vectors.
Now $v_3$ can be any vector which is not an linear combination
of $v_1$ and $v_2$. There are $q^n-q^2$ such vectors.
Continuing in this way we find that
\[\left|\mathsf{GL}_n\left(F\right)\right|
=\left(q^n-1\right)\left(q^n-q\right)\cdots\left(q^n-q^{n-1}\right).\]
Now to count the elements of $\mathsf{SL}_n\left(F\right)$
we regard $\mathsf{GL}_n\left(F\right)$ as a group
and $\det:\mathsf{GL}_n\left(F\right)\to\mathsf{GL}_1\left(F\right)$
as a group homomorphism. Then $\mathsf{SL}_n\left(F\right)
=\ker\left(\det\right)$ is a normal subgroup of $\mathsf{GL}_n\left(F\right)$
and $\mathsf{GL}_n\left(F\right)/\ker\left(\det\right)$
is isomorphic to the image
of $\det$. Since $\det$ is surjective we have
$\frac{\left|\mathsf{GL}_n\left(F\right)\right|}
{\left|\mathsf{SL}_n\left(F\right)\right|}=q-1$
so that
\[\left|\mathsf{GL}_n\left(F\right)\right|
=\left(q-1\right)^{-1}
\left(q^n-1\right)\left(q^n-q\right)\cdots\left(q^n-q^{n-1}\right).\]

\item\label{Number7} %7
If $f=x_0^{i_0}\cdots x_n^{i_n}$ is a monomial with $i_0+\cdots+i_n=m$ then 
$\frac{\partial f}{\partial x_j}=i_jx_0^{i_0}\cdots x_j^{i_j-1}
\cdots x_n^{i_n}$ so that
\[\sum_{j=0}^n
x_j\frac{\partial f}{\partial x_j}
=\sum_{j=0}^ni_jx_0^{i_0}\cdots x_n^{i_n}
=f\sum_{j=0}^ni_j=mf.\]
It follows that if $f=\sum_{i_0,\ldots,i_n}
\alpha_{i_0,\ldots,i_n}x_0^{i_0}\cdots x_n^{i_n}$ then
\begin{multline*}
\sum_{j=0}^n
x_j\frac{\partial f}{\partial x_j}
=\sum_{j=0}^n x_j\frac{\partial}{\partial x_j}
\sum_{i_0,\ldots,i_n}
\alpha_{i_0,\ldots,i_n}x_0^{i_0}\cdots x_n^{i_n}
=\sum_{i_0,\ldots,i_n}\alpha_{i_0,\ldots,i_n}
\sum_{j=0}^n x_j\frac{\partial}{\partial x_j}
x_0^{i_0}\cdots x_n^{i_n}\\
=\sum_{i_0,\ldots,i_n}
\alpha_{i_0,\ldots,i_n}
mx_0^{i_0}\cdots x_n^{i_n}=mf.
\end{multline*}

\item %8
$mf\left(\overline{a}\right)=
\sum_{j=0}^n x_j\frac{\partial f}{\partial x_j}\left(\overline{a}\right)
=0$ by~\autoref{Number7}.
But since $m$ is prime to the characteristic
of~$F$ it follows that $f\left(\overline{a}\right)=0$.

\item %9
$\frac{\partial}{\partial x_j}
\left(a_0x_0^m+\cdots+a_nx_n^m\right)=
mx_j^{m-1}$. Thus if $\left[x_0,\ldots,x_n\right]$
is a zero of all the partials
then $0=mx_j^{m-1}$ for all~$j$. But since $m$ is prime
to the characteristic of $F$ it follows that $x_j=0$ for all~$j$.
However, $\left[0,\ldots,0\right]$ is not a point of
$\mathbb{P}^n$ so $\overline{H}_{\overline{f}}$ has no singular points.

\item %10
\item %11
Let $f=y^2-x^3\in F\left[x,y\right]$.
Then the homogenization of $f$
is $\overline{f}=ty^2-x^3\in F\left[t,x,y\right]$.
Then $\nabla\overline{f}=\left(y^2,-3x^2,2yt\right)$ so that
$\nabla\overline{f}\left(1,0,0\right)
=\left(0,0,0\right)$. This shows that $\left(1,0,0\right)$,
the point on the projective closure
of $H_f$ corresponding with the origin, is a singular point.
Therefore the origin is a singular point of $H_f$.

\item %12
Let $f=x^2+y^2+x^2y^2$. Then the homogenization of $f$
is $\overline{f}=t^2x^2+t^2y^2+x^2y^2$. The points at infinity
of $\overline{H}_{\overline{f}}$ are the points $\left[0,x,y\right]$
satisfying $0=\overline{f}\left(0,x,y\right)=x^2y^2$.
It follows that either $x=0$ or $y=0$. If $x=0$ then $y$
can take any value. However, as $y$ ranges over $F$ the points
$\left[0,0,y\right]$ all correspond with the same point
$\left[0,0,1\right]$. Similarly, if $y=0$ we find only
the point $\left[0,1,0\right]$. Thus the points at infinity are
$\left[0,1,0\right]$ and $\left[0,0,1\right]$.
We calculate $\nabla\overline{f}
=\left(2tx^2+2ty^2,2t^2x+2xy^2,2t^2y+2x^2y\right)$.
Substituting each point at infinity gives $\left(0,0,0\right)$
so both points are singular.

\item %13
Let $f=ax^2+bxy+cy^2-1$. Then the homogenization of $f$
is $\overline{f}=ax^2+bxy+cy^2-t^2$. Now if
$\overline{a}=\left[0,x,y\right]$ is a point at infinity
then $0=\overline{f}\left(\overline{a}\right)=ax^2+bxy+cy^2$ tells us
that either $x$ or $y$ must be nonzero, say $y\ne 0$. Then
\[x=\frac{-by\pm\sqrt{b^2y^2-4acy^2}}{2a}
=\frac{-by\pm y\sqrt{b^2-4ac}}{2a}.\]
If $b^2-4ac$ is not a square in $F$ then there are no solutions
$x$ and therefore no points at infinity.
If $b^2-4ac$ is a square then there are either
one or two points at infinity, depending on whether $b^2-4ac$
is zero or not. In the case $b^2-4ac=0$ the point
is $\overline{a}=\left[0,-\frac{by}{2a},y\right]$.
Note that this expression does not depend on the representative $y$.
We calculate $\nabla\overline{f}=\left(
-2t,2ax+by,bx+2cy\right)$ so that
$\nabla\overline{f}\left(\overline{a}\right)=\left(
0,0,0\right)$ since $0=b^2-4ac$.
This shows that $\overline{a}$ is singular.

\item %14
Let $f=x^3-y^2+ax+b$. Then the homogenization of $f$
is $\overline{f}=x^3-ty^2+at^2x+bt^3$. We calculate
$\nabla\overline{f}=\left(-y^2+2atx+3bt^2,3x^2+at^2,-2ty\right)$.
If $\left[0,x,y\right]$ is singular then
$\left(0,0,0\right)=\left(-y^2,3x^2,0\right)$.
This means that $x=y=0$ so there are no singular points at infinity.
If $\left[1,x,y\right]$ is singular then
$\left(0,0,0\right)=\left(-y^2+2ax+3b,3x^2+a,-2y\right)$.
Equating the last coordinates gives $y=0$ so that
\begin{equation}\label{Equation14}
\left(0,0\right)=\left(2ax+3b,3x^2+a\right).
\end{equation}
Note that if $a$ were zero then \autoref{Equation14}
would tell us that $b$ is also zero, contrary to the
assumption that $4a^3+27b^2\ne 0$. Therefore $a\ne 0$.
Equating the first coordinates of \autoref{Equation14}
gives $x=-\frac{3b}{2a}$.
Now equating the second coordinates of \autoref{Equation14}
gives $0=3\left(-\frac{3b}{2a}\right)^2+a=\frac{27b^2}{4a^2}+a$
Clearing fractions gives $0=27b^2+4a^3$ contrary to the
assumption that $4a^3+27b^2\ne 0$. Therefore there are no
affine singular points.
\end{enumerate}
\end{document}
