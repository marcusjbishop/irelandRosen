\documentclass[12pt]{article}
\usepackage{multicol,graphicx}
\usepackage[colorlinks,breaklinks,linkcolor=red,citecolor=blue]
{hyperref} 
\def\sectionautorefname~#1\null{\S#1\null}
\usepackage{charter,amsmath,amssymb,breakurl}
\def\equationautorefname~#1\null{(#1)\null}
\def\itemautorefname~#1\null{(#1)\null}
\usepackage{eulervm}
\usepackage[letterpaper,margin=.75in]{geometry}
\renewcommand{\pmod}[1]{\left(\mathsf{mod}\;#1\right)}
\title{Chapter 7}
\renewcommand{\theenumi}{7.\arabic{enumi}}
\author{}\date{}
\let\ker\relax\DeclareMathOperator{\ker}{\mathsf{ker}}
\let\ord\relax\DeclareMathOperator{\ord}{\mathsf{ord}}
\let\ln\relax\DeclareMathOperator{\ln}{\mathsf{ln}}
\let\deg\relax\DeclareMathOperator{\deg}{\mathsf{deg}}
\let\sin\relax\DeclareMathOperator{\sin}{\mathsf{sin}}
\let\arctan\relax\DeclareMathOperator{\arctan}{\mathsf{arctan}}
\let\cos\relax\DeclareMathOperator{\cos}{\mathsf{cos}}
\let\sec\relax\DeclareMathOperator{\sec}{\mathsf{sec}}
\let\min\relax\DeclareMathOperator*{\min}{\mathsf{min}}
\let\max\relax\DeclareMathOperator*{\max}{\mathsf{max}}
\let\sup\relax\DeclareMathOperator*{\sup}{\mathsf{sup}}
\let\inf\relax\DeclareMathOperator*{\inf}{\mathsf{inf}}
\let\lim\relax\DeclareMathOperator*{\lim}{\mathsf{lim}}
\everymath{\displaystyle}

\makeatletter
\def\legendre@dash#1#2{\hb@xt@#1{%
  \kern-#2\p@
  \cleaders\hbox{\kern.5\p@
    \vrule\@height.2\p@\@depth.2\p@\@width\p@
    \kern.5\p@}\hfil
  \kern-#2\p@
  }}
\def\@legendre#1#2#3#4#5{\mathopen{}\left(
  \sbox\z@{$\genfrac{}{}{0pt}{#1}{#3#4}{#3#5}$}%
  \dimen@=\wd\z@
  \kern-\p@\vcenter{\box0}\kern-\dimen@\vcenter{\legendre@dash\dimen@{#2}}\kern-\p@
  \right)\mathclose{}}
\newcommand\legendre[2]{\mathchoice
  {\@legendre{0}{1}{}{#1}{#2}}
  {\@legendre{1}{.5}{\vphantom{1}}{#1}{#2}}
  {\@legendre{2}{0}{\vphantom{1}}{#1}{#2}}
  {\@legendre{3}{0}{\vphantom{1}}{#1}{#2}}
}
\def\dlegendre{\@legendre{0}{1}{}}
\def\tlegendre{\@legendre{1}{0.5}{\vphantom{1}}}
\makeatother
\begin{document}
\maketitle

\begin{enumerate}
\item %1

Let $\lambda$ be a generator of the character group
defined by $\lambda\left(g^j\right)=\zeta^j$
where $g$ is a primitive root~$\pmod{p}$
and $\zeta=e^{\frac{2\pi i}{p}}$.
Note that $\lambda$ is injective.
Then taking $\chi=\lambda^{\frac{p-1}{d}}$ the characters
of order dividing $d$ are
$\chi,\chi^2,\chi^3,\ldots,\chi^{d-1},\epsilon=\chi^d$.

If $a\equiv 0\pmod{p}$ then $x^m\equiv a$ has
only one solution, namely $x\equiv 0$.
Observe that $\epsilon\left(0\right)=1$ while
$\chi^j\left(0\right)=0$ for all $1\le j\le d-1$.
Thus $\sum_{j=0}^{d-1}\chi^j\left(0\right)=1$.

Otherwise assume that $a\not\equiv 0$.
Then $x^m\equiv a$ has either $d$~solutions
in case $a^{\frac{p-1}{d}}\equiv 1\pmod{p}$ or
zero solutions in case $a^{\frac{p-1}{d}}\not\equiv 1\pmod{p}$.

Suppose that $x^m\equiv a$ has solutions and let $b\in\mathbb{Z}$
be such that $b^m\equiv a$.
Then $\chi\left(a\right)=\chi\left(b\right)^m=1$ since
$d=\left(d,p-1\right)$ is a divisor of $m$. Thus
$\sum_{j=0}^{d-1}\chi^j\left(a\right)=d$.

Suppose that $x^m\equiv a$ has no solutions.
Then $\chi\left(a\right)$ is a $d$th root of unity.
However, by the observation above that $\chi$
is injective $\chi\left(a\right)
=\lambda\left(a^{\frac{p-1}{d}}\right)\ne 1$
since $a^{\frac{p-1}{d}}\ne 1$.
Thus $\sum_{j=0}^{d-1}\chi^j\left(a\right)=0$.

\item %2
Note that $d=\left(m,p-1\right)$ is a divisor
of $p-1$ so that $\left(d,p-1\right)=d$.
It follows that if $a\not\equiv 0$ and
$x^d=a$ has solutions, then it has $d$~solutions.
In fact, the bijection $\left\{x\mid x^m=a\right\}\to
\left\{x\mid x^d=a\right\}$ is given by $x\mapsto x^q$
where $q\in\mathbb{Z}$ is such that $m=dq$.
This bijection extends to a bijection
\[\bigcup_{\substack{\left(y_1,y_2,\ldots,y_k\right)\\
\sum_{i=1}^ka_iy^{m_i}=b}}
\bigcap_{i=1}^k\left\{x\mid x^{m_i}=y_i\right\}
\longrightarrow\bigcup_{\substack{\left(y_1,y_2,\ldots,y_k\right)\\
\sum_{i=1}^ka_iy^{d_i}=b}}
\bigcap_{i=1}^k\left\{x\mid x^{m_i}=y_i\right\}\]
from the set of solutions of $\sum_{i=1}^ka_ix^{m_i}=b$
to the set of solutions of $\sum_{i=1}^ka_ix^{d_i}=b$.

\item\label{Exercise3} %3
Since $N\left(x^2=a\right)=1+\rho\left(a\right)$
we have $\rho\left(a\right)=1-N\left(x^2=a\right)=\tlegendre{a}{p}$.
It follows that
\[J\left(\chi,\rho\right)=\sum_{a+b=0}\chi\left(a\right)\rho\left(b\right)
=\sum_{a=0}^{p-1}\chi\left(1-a\right)\rho\left(a\right)
=\sum_{a=0}^{p-1}\chi\left(1-a\right)\dlegendre{a}{p}.\]
Now observe that
\[0=\sum_{a=0}^{p-1}\chi\left(a-1\right)
=\sum_{\text{$t$ a residue}}\chi\left(t-1\right)
+\sum_{\text{$t$ a nonresidue}}\chi\left(t-1\right)\]
so that
\[\sum_{\text{$t$ a nonresidue}}\chi\left(t-1\right)
=-\sum_{\text{$t$ a residue}}\chi\left(t-1\right)
=-\sum_{t=1}^{\frac{p-1}{2}}\chi\left(t-1\right).\]
It follows that
\[J\left(\chi,\rho\right)
=\sum_{a=0}^{p-1}\chi\left(1-a\right)\dlegendre{a}{p}
=\sum_{\text{$t$ a residue}}\chi\left(t-1\right)
-\sum_{\text{$t$ a nonresidue}}\chi\left(t-1\right)
=\sum_{t=0}^{p-1}\chi\left(1-t^2\right).\]

\item\label{Exercise4} %4
\begin{align*}
\sum_t\chi\left(t\right)\chi\left(k-t\right)
&=\sum_t\chi\left(kt\right)\chi\left(k-kt\right)\\
&=\chi\left(k^2\right)\sum_t\chi\left(t\right)\chi\left(1-t\right)\\
&=\chi\left(\frac{k^2}{2^2}\right)
\sum_t\chi\left(2t\right)\chi\left(2-2t\right)\\
&=\chi\left(\frac{k^2}{2^2}\right)
\sum_t\chi\left(t\right)\chi\left(2-t\right)\\
&=\chi\left(\frac{k^2}{2^2}\right)
\sum_s\chi\left(1+s\right)\chi\left(1-s\right)
\qquad\text{(where $t=1+s$)}\\
&=\chi\left(\frac{k^2}{2^2}\right)
\sum_s\chi\left(1-s^2\right)\\
&=\chi\left(\frac{k^2}{2^2}\right)J\left(\chi,\rho\right)
\qquad\text{(by \autoref{Exercise3})}
\end{align*}

\item %5
\begin{align*}
\left(g\left(\chi\right)\right)^2
&=\left(\sum\chi\left(t\right)\zeta^t\right)^2\\
&=\sum_{t=0}^{p-1}\zeta^t
\sum_{j+k\equiv t}\chi\left(j\right)\chi\left(k\right)\\
&=\sum_{t=0}^{p-1}\zeta^t
\sum_{j=0}^{p-1}\chi\left(j\right)\chi\left(t-j\right)\\
&=\sum_{t=0}^{p-1}\zeta^t\chi\left(\frac{t^2}{2^2}\right)
J\left(\chi,\rho\right)\qquad(\text{by \autoref{Exercise4}})\\
&=\chi\left(2\right)^{-2}J\left(\chi,\rho\right)
\sum_{t=0}^{p-1}\chi\left(t^2\right)\zeta^t\\
&=\chi\left(2\right)^{-2}J\left(\chi,\rho\right)g\left(\chi^2\right)
\end{align*}

\item %6
\begin{align*}
J\left(\chi,\rho\right)
&=\sum_t\chi\left(1+t\right)\chi\left(1-t\right)
\qquad\text{(by \autoref{Exercise3})}\\
&=\sum_s\chi\left(s\right)\chi\left(2-s\right)
\qquad\text{(where $s=1+t$)}\\
&=\sum_s\chi\left(2s\right)\chi\left(2-2s\right)\\
&=\chi\left(2\right)^2
\sum_s\chi\left(s\right)\chi\left(1-s\right)\\
&=\chi\left(2\right)^2J\left(\chi,\chi\right)
\end{align*}

\end{enumerate}
\end{document}

