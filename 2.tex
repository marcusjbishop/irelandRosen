\documentclass[12pt]{article}
\usepackage{multicol,graphicx}
\usepackage[colorlinks,breaklinks,linkcolor=red,citecolor=blue]
{hyperref} 
\def\sectionautorefname~#1\null{\S#1\null}
\usepackage{charter,amsmath,amssymb,breakurl}
\def\equationautorefname~#1\null{(#1)\null}
\def\itemautorefname~#1\null{(#1)\null}
\usepackage{eulervm}
\usepackage[letterpaper,margin=.75in]{geometry}
\renewcommand{\pmod}[1]{\left(\mathsf{mod}\;#1\right)}
\title{Chapter 2}
\author{}\date{}
\let\ord\relax\DeclareMathOperator{\ord}{\mathsf{ord}}
\let\ln\relax\DeclareMathOperator{\ln}{\mathsf{ln}}
\let\deg\relax\DeclareMathOperator{\deg}{\mathsf{deg}}
\let\sin\relax\DeclareMathOperator{\sin}{\mathsf{sin}}
\let\arctan\relax\DeclareMathOperator{\arctan}{\mathsf{arctan}}
\let\cos\relax\DeclareMathOperator{\cos}{\mathsf{cos}}
\let\sec\relax\DeclareMathOperator{\sec}{\mathsf{sec}}
\let\min\relax\DeclareMathOperator*{\min}{\mathsf{min}}
\let\max\relax\DeclareMathOperator*{\max}{\mathsf{max}}
\let\sup\relax\DeclareMathOperator*{\sup}{\mathsf{sup}}
\let\inf\relax\DeclareMathOperator*{\inf}{\mathsf{inf}}
\let\lim\relax\DeclareMathOperator*{\lim}{\mathsf{lim}}
\everymath{\displaystyle}
\renewcommand{\theenumi}{2.\arabic{enumi}}
\begin{document}
\maketitle
\thispagestyle{empty}

\begin{enumerate}
\item %1
Suppose that $m_1\left(x\right),m_2\left(x\right),\ldots,m_N\left(x\right)$
is a complete list of monic irreducible elements of $k\left[x\right]$
and fix an even number $D>0$. Now if $f\left(x\right)\in k\left[x\right]$
is any polynomial of degree less than $D$ then we can write
$f\left(x\right)=g\left(x\right)h\left(x\right)^2$
for some $g\left(x\right),h\left(x\right)\in k\left[x\right]$
with $h\left(x\right)$ squarefree. Note that since
\[D=\deg f\left(x\right)\ge\deg h\left(x\right)^2\]
it follows that $\deg h\left(x\right)\le D/2$.
Note also that since $g\left(x\right)$ is squarefree, we can write
\[g\left(x\right)=\alpha m_1\left(x\right)^{\epsilon_1}
m_2\left(x\right)^{\epsilon_2}\cdots
m_N\left(x\right)^{\epsilon_N}\]
for some unit $\alpha\in k$ and $\epsilon_i=0,1$ for all
$1\le i\le N$. There are $\left|k\right|2^N$ possible combinations
of $\alpha,\epsilon_1,\ldots,\epsilon_N$.

Observe that there are $\left|k\right|^{D}$ elements of $k\left[x\right]$
of degree less than $D$.
Similarly there are $\left|k\right|^{D/2}$ elements
of degree less than $D/2$. It follows from the observations above that
\[\left|k\right|^D\le\left|k\right|^{D/2}\cdot\left|k\right|2^N
\qquad\iff\qquad\left|k\right|^{D/2-1}\le 2^N\]
which is false for large enough $D$. 

\item %2
Put \[S=\left\{\left.\frac{a}{b}\right|\text{
$\ord_{p_i}{a}\ge\ord_{p_i}{b}$ for all $1\le i\le t$}\right\}.\]
Suppose $\frac{a}{b},\frac{c}{d}\in S$. 
Then abbreviating $\ord_{p_i}$ to $\ord$ we have
\[\ord\left(ad+bc\right)
\ge\min\left\{\ord\left(ad\right),
\ord\left(bc\right)\right\}
=\min\left\{\ord\left(a\right)\ord\left(d\right),
\ord\left(b\right)\ord\left(c\right)\right\}
\ge\ord\left(b\right)\ord\left(d\right)\]
since both elements of
$\left\{\ord\left(a\right)\ord\left(d\right),
\ord\left(b\right)\ord\left(c\right)\right\}$ are greater
than or equal to $\ord\left(b\right)\ord\left(d\right)$ by the assumption
$\frac{a}{b},\frac{c}{d}\in S$. 
It follows that $\frac{a}{b}+\frac{c}{d}
=\frac{ad+bc}{bd}\in S$.

\item %3 
If $p_1,p_2,\ldots,p_t$ were the only primes, then
put $n=p_1p_2\cdots p_t$. Now any integer
$2<i<n$ has a prime divisor, which by assumption
must be one of $p_1,p_2,\ldots,p_t$. It follows that
$\left(i,n\right)\ne 1$ for any $2<i<n$ so that
$\varphi\left(n\right)=1$. But this is in contradiction with
\[\varphi\left(n\right)=
n\left(1-\frac{1}{p_1}\right)
\left(1-\frac{1}{p_2}\right)
\cdots\left(1-\frac{1}{p_t}\right)\]
unless $n=2$, a case that we rule out, knowing that
$2$ is not the only prime.

\item\label{Mersenne} %4
The factorization
\[\left(a^{2^n}-1\right)
=\left(a^{2^m}+1\right)\left(a^{2^n-2^m}
-a^{2^n-2\left(2^m\right)}+a^{2^n-3\left(2^m\right)}
-\cdots+a^{2^n-\left(2^{n-m}-1\right)2^m}-1\right)\]
shows that if $p$ is an odd prime dividing $a^{2^m}+1$
then $p$ also divides $a^{2^n}-1$.
It follows that $p$ cannot be a factor of both $a^{2^m}+1$
and $a^{2^n}+1$ since then $p$ also divides
$2^{2^n}+1-\left(2^{2^n}-1\right)=2$, a contradiction.
Finally, if $a$ is odd, then $a^{2^n}+1$ and $a^{2^m}+1$
are both even, so $2$ is their only common prime factor.
If $a$ is even, then $a^{2^n}+1$ and $a^{2^m}+1$
are both odd and thereby have no prime factors in common.

\item %5
Taking $a=2$ in \autoref{Mersenne} we define the sequence
$x_k=2^{2^k}+1$ for $k\ge 0$.
Note that by \autoref{Mersenne} $\left(x_j,x_k\right)=1$
for any $j\ne k$. It follows that if $p_k$ is any prime
factor of $x_k$, then $p_k$ is a sequence of distinct primes.

\item %6
There must be an elegant solution to this.
Each $\left[\frac{n}{p^j}\right]$ contributes one
to each multiple of $p^j$ appearing as a factor
of $n!$. So
$\sum_{j\ge 1}\left[\frac{n}{p^j}\right]$ is the same
as $\sum_{i=1}^n\ord_p{i}=\ord_p{n!}$.

\item %7
$\ord_p{n!}=\sum_{j\ge 1}\left[\frac{n}{p^j}\right]
\le \sum_{j\ge 1}\frac{n}{p^j}
=n\left(\frac{\frac{1}{p}}{1-\frac{1}{p}}\right)
=\frac{n}{p-1}$.
Now since $N=\prod_{p\mid N}p^{\ord_p{N}}$ for any integer
$N>0$ we have
\begin{equation}\label{OrdApproximation}
\left(n!\right)^{1/n}=\left(\prod_{p|n!}p^{\ord_p{n!}}\right)^{1/n}
=\prod_{p|n!}p^{\frac{1}{n}\ord_p{n!}}
\le\prod_{p|n!}p^{\frac{1}{p-1}}
\end{equation}

\item %8
The lefthand side of \autoref{OrdApproximation} goes to
$\infty$ as $n\to\infty$. However, the righthand side only
takes finitely many values if there were only finitely many
primes.

\item\label{Determined} %9
If $p_1^{\alpha_1}p_2^{\alpha_2}\cdots p_k^{\alpha_k}$
is the prime factorization of $n$, then
\[f\left(n\right)=f\left(p_1^{\alpha_1}\right)
f\left(p_2^{\alpha_2}\right)\cdots f\left(p_k^{\alpha_k}\right)\]
since the factors
$p_1^{\alpha_1},p_2^{\alpha_2},\ldots, p_k^{\alpha_k}$
are relatively prime.

\item\label{GMultiplicative} %10
Suppose $\left(a,b\right)=1$. Now if $d|ab$ then we can
write $d=d_1d_2$ where $d_1|a$ and $d_2|b$. It follows that
$\left(d_1,d_2\right)=1$. Thus
\[g\left(ab\right)=\sum_{d|ab}f\left(d\right)
=\sum_{\substack{d_1|a\\d_2|b}}f\left(d_1d_2\right)
=\sum_{d_1|a}f\left(d_1\right)\sum_{d_2|b}f\left(d_2\right)
=g\left(a\right)g\left(b\right).\]

\item\label{MuMultiplicative} %11
Put $f\left(n\right)=\frac{\mu\left(n\right)}{n}$ for $n\in\mathbb{N}$
and suppose $\left(a,b\right)=1$.
If $a=b=1$ then $f\left(ab\right)=1=f\left(a\right)f\left(b\right)$.
If one of $a,b$ is not squarefree, then $ab$ is not squarefree
(and in fact the converse holds, since $\left(a,b\right)=1$)
so that $f\left(ab\right)=0=f\left(a\right)f\left(b\right)$.
Finally, if $a=p_1p_2\cdots p_k$ and $b=q_1q_2\ldots q_l$
where $p_1,p_2,\ldots,p_k,q_1,q_2,\ldots,q_l$ are distinct
primes, then
\[f\left(ab\right)=\frac{\left(-1\right)^{k+l}}{ab}
=\frac{\left(-1\right)^k}{a}\frac{\left(-1\right)^l}{b}
=f\left(a\right)f\left(b\right).\]
This shows that $f$ is multiplicative.
Then $g\left(n\right)=\sum_{d|n}
\frac{\mu\left(d\right)}{d}$ is also multiplicative
by \autoref{GMultiplicative}. 
Putting $h\left(n\right)=\frac{\varphi\left(n\right)}{n}$
we observe that
\[g\left(p^k\right)=1-\frac{1}{p}=h\left(p^k\right).\] 
In other words, $g$ and $h$ agree on prime powers.
But since $g$ is determined by its values
on prime powers by \autoref{Determined}, this means that $g=h$.
It follows that
$\sum_{d|n}\frac{\mu\left(d\right)}{d}
=\frac{\varphi\left(n\right)}{n}$
or $\varphi\left(n\right)
=n\sum_{d|n}\frac{\mu\left(d\right)}{d}$.

\item %12
\item %13
If $\left(a,b\right)=1$ then
\[\sigma_k\left(a,b\right)
=\sum_{d\mid ab}d^k=\sum_{d_1\mid a}\sum_{d_2\mid b}
\left(d_1d_2\right)^k
=\sum_{d_1\mid a}d_1^k\sum_{d_2\mid b}d_2^k
=\sigma_k\left(a\right)\sigma_k\left(b\right)\]
shows that $\sigma_k$ is multiplicative.
Now if $p_1^{\alpha_1}\cdots p_t^{\alpha_t}$
is the prime factorization of $n$ then
\begin{align*}
\sigma_k\left(n\right)
&=\left(1^k+p_1^k+\cdots p_1^{k\alpha_1}\right)
\cdots\left(1^k+p_t^k+\cdots p_t^{k\alpha_t}\right)\\
&=\left(\frac{p_1^{k\left(\alpha_1+1\right)}-1}{p_1^k-1}\right)
\cdots\left(\frac{p_1^{k\left(\alpha_t+1\right)}-1}{p_t^k-1}\right)
\end{align*}

\item %14
Knowing that $\mu$ is multiplicative by the argument
in \autoref{MuMultiplicative} we know that
\begin{align*}
h\left(a,b\right)
&=\sum_{d_1\mid a}\sum_{d_2\mid b}
\mu\left(\frac{ab}{d_1d_2}\right)f\left(d_1d_2\right)\\
&=\sum_{d_1\mid a}
\mu\left(\frac{a}{d_1}\right)f\left(d_1\right)
\sum_{d_2\mid b}
\mu\left(\frac{b}{d_2}\right)f\left(d_2\right)\\
&=h\left(a\right)h\left(b\right)
\end{align*}
if $\left(a,b\right)=1$.

\item %15
Knowing that $\sum_{d\mid n}1=v\left(n\right)$
we have $\sum_{d\mid n}\mu\left(\frac{n}{d}\right)v\left(d\right)=1$
by M\"obius inversion.
Similarly, knowing that $\sum_{d\mid n}d=\sigma\left(n\right)$
we have $\sum_{d\mid n}\mu\left(\frac{n}{d}\right)\sigma\left(d\right)=n$.

\item %16
If $p_1^{\alpha_1}p_2^{\alpha_2}\cdots p_k^{\alpha_k}$
is the prime factorization of $n$, then
$v\left(n\right)=\prod_{j=1}^k\left(\alpha_j+1\right)$
is odd if and only if $\alpha_1,\alpha_2,\ldots,\alpha_k$
are all even, if and only if $n$ is a square.

\item %17
If $p_1^{\alpha_1}p_2^{\alpha_2}\cdots p_k^{\alpha_k}$
is the prime factorization of $n$, then
we observe that if $p_i$ is an odd prime factor of $n$ then
$1+p_i+p_i^2+\cdots+p_i^{\alpha_i}$ is odd if and only
if the number of terms
$\alpha_i+1$ is odd, if and only if $\alpha_i$ is even.
If $2$ is a prime factor of $n$, say $p_1=2$ then
$1+2+2^2+\cdots +2^{\alpha_1}$ is odd for any value of $\alpha_1$.
It follows that
\[\sigma\left(n\right)=\left(1+p_1+\cdots+p_1^{\alpha_1}\right)
\cdots\left(1+p_k+\cdots+p_k^{\alpha_k}\right)\]
is odd if and only if
all the $\alpha_i$ corresponding with odd primes $p_i$
are odd, which happens when $n$ is a square or twice a square.

\end{enumerate}
\end{document}
