\documentclass[12pt]{article}
\usepackage{multicol,graphicx}
\usepackage[colorlinks,breaklinks,linkcolor=red,citecolor=blue]
{hyperref} 
\def\sectionautorefname~#1\null{\S#1\null}
\usepackage{charter,amsmath,amssymb,breakurl}
\def\equationautorefname~#1\null{(#1)\null}
\def\itemautorefname~#1\null{(#1)\null}
\usepackage{eulervm}
\usepackage[letterpaper,margin=.75in]{geometry}
\renewcommand{\pmod}[1]{\left(\mathsf{mod}\;#1\right)}
\title{Chapter 6}
\renewcommand{\theenumi}{6.\arabic{enumi}}
\author{}\date{}
\let\ord\relax\DeclareMathOperator{\ord}{\mathsf{ord}}
\let\ln\relax\DeclareMathOperator{\ln}{\mathsf{ln}}
\let\deg\relax\DeclareMathOperator{\deg}{\mathsf{deg}}
\let\sin\relax\DeclareMathOperator{\sin}{\mathsf{sin}}
\let\arctan\relax\DeclareMathOperator{\arctan}{\mathsf{arctan}}
\let\cos\relax\DeclareMathOperator{\cos}{\mathsf{cos}}
\let\sec\relax\DeclareMathOperator{\sec}{\mathsf{sec}}
\let\min\relax\DeclareMathOperator*{\min}{\mathsf{min}}
\let\max\relax\DeclareMathOperator*{\max}{\mathsf{max}}
\let\sup\relax\DeclareMathOperator*{\sup}{\mathsf{sup}}
\let\inf\relax\DeclareMathOperator*{\inf}{\mathsf{inf}}
\let\lim\relax\DeclareMathOperator*{\lim}{\mathsf{lim}}
\everymath{\displaystyle}
\begin{document}
\maketitle

\begin{enumerate}
\item %1
Write $\alpha=\sqrt{2}+\sqrt{3}$.
We calculate $\alpha^2=5+2\sqrt{6}$
and $\alpha^4=49+10\sqrt{6}$.
It follows that $\alpha$ is a root of
$x^4-10x^2+1$, so $\alpha$ is an algebraic integer.

\item %2
Suppose
$x^n+a_{n-1}x^{n-1}+\cdots+a_0$ is the
minimal polynomial of $\alpha$.
Evaluating at $x=\alpha$ and multiplying by $d^n$
where $d$ is a common multiple of the denominators
of $a_0,a_1,\ldots,a_{n-1}$ gives
\[0=d^n\alpha^n+a_{n-1}\alpha^{n-1}d^n+\cdots+a_0d^n
=\left(d\alpha\right)^n+a_{n-1}d\left(d\alpha\right)^{n-1}
+\cdots+a_0d^n.\]
This shows that $d\alpha$ is a root of the monic polynomial
$x^n+a_{n-1}dx^{n-1}+\cdots+a_0d^n$
so that $d\alpha$ is an algebraic integer.

\item %3
Let $V\subseteq\mathbb{C}$ be the $\mathbb{Q}$-linear span of
\begin{equation}\label{Dependence}
\left\{\alpha^i\beta^j x^k\mid 0\le i<m,0\le j<n,0\le k<2\right\}
\end{equation}
where $m=\det{\alpha}$ and $n=\deg{\beta}$.
We need to check that $x\left(\alpha^i\beta^j x^k\right)$ lies in $V$
where $i,j,k$ lie in the ranges shown in \autoref{Dependence}.
If $k=0$ then $\alpha^i\beta^j x$ is one of the monomials in
\autoref{Dependence} However, if $k=1$ then we write
\[\alpha^i\beta^j x^2=\alpha^i\beta^j\left(-\alpha x-\beta\right)
=-\alpha^{i+1}\beta^jx-\alpha^i\beta^{j+1}x.\]
Now if either $m=i+1$ or $n=j+1$ then we use the minimal polynomials
of $\alpha$ or $\beta$ in the same way to write
$\alpha^i\beta^jx^2$ as a linear combination of elements of
\autoref{Dependence}.

\item %4
Suppose that $f\left(x\right)=a_mx^m+a_{m-1}x^{m-1}+\cdots+a_0$
and $g\left(x\right)=b_nx^n+b_{n-1}x^{n-1}+\cdots+b_0$
are primitive polynomials and let $p$ be any prime.
By minimality of $f$ the set of coefficients of $f$ not divisisble by $p$
is nonempty, and similarly for $g$.
Let $i$ be minimal such that $p\nmid a_i$ and let
$j$ be minimal such that $p\nmid b_j$.
Then the coefficient of $x^{i+j}$ in $f\left(x\right)g\left(x\right)$
is $a_ib_j$ plus other terms $a_kb_l$ with either $k<i$ or $l<j$.
But if $k<i$ or $l<j$ then $p$ divides $a_kb_l$ by minimality
of $i$ and $j$. So $p$ divides all the terms of the coefficient of $x^{i+j}$
except $a_ib_j$. We conclude that $p$ does {\em not} divide this coefficient.
This means that the coefficients of $f\left(x\right)g\left(x\right)$
are relatively prime, for if not, then there would be a prime $p$
dividing them all.

\item %5
Since $\alpha$ is an algebraic integer there exists
a monic polynomial $g\left(x\right)\in\mathbb{Z}\left[x\right]$
with $g\left(\alpha\right)=0$.
The monic polynomial $f\left(x\right)\in\mathbb{Q}\left[x\right]$
of {\em minimal} degree such that $f\left(x\right)=0$ might fail to
have integer coefficients. Let $d$ be the least common multiple
of the denominators of the coefficients of $f$. Then
$df\left(x\right)\in\mathbb{Z}\left[x\right]$.
Then $df\left(x\right)$ divides $g\left(x\right)$
so there exists $h\left(x\right)\in\mathbb{Z}\left[x\right]$
such that $df\left(x\right)h\left(x\right)=g\left(x\right)$.
But $d$ divides the leading coefficient of $df\left(x\right)h\left(x\right)$
while the leading coefficient of $g\left(x\right)h\left(x\right)$
is $1$. It follows that $d=1$ so that
$f\left(x\right)\in\mathbb{Z}\left[x\right]$.

\item %6
Put $R=\left\{r+s\alpha\mid r,s\in\mathbb{Q}\right\}$.
Naturally $\mathbb{Q}\left[\alpha\right]\subseteq R$.
Conversely, we observe
that since $\alpha$ satisfies $x^2+mx+n$ we know that
$\alpha^2=-m\alpha-n$. This observation can be used to reduce
any polynomial in $\alpha$ to one of the form $r+s\alpha$.
This shows that $R=\mathbb{Q}\left[\alpha\right]$.

Suppose $r_1+s_1\alpha,r_2+s_2\alpha\in R$.
Then $\left(r_1+s_1\alpha\right)-\left(r_2+s_2\alpha\right)
=\left(r_1-r_2\right)+\left(s_1-s_2\right)\alpha\in R$ and
\[\left(r_1+s_1\alpha\right)\left(r_2+s_2\alpha\right)
=r_1r_2+\left(r_1s_2+r_2s_1\right)\alpha+s_1s_2\alpha^2
=r_1r_2-s_1s_2n+\left(r_1s_2+r_2s_1-s_1s_2m\right)\alpha\]
shows that $R$ is a ring.

Since $\alpha$ satisfies $x^2+mx+n$ we know that
$\alpha=\frac{-m\pm\sqrt{m^2-4n}}{2}=\frac{-m\pm D_0\sqrt{D}}{2}$.
Thus $\alpha\in\mathbb{Z}\left[\sqrt{D}\right]$ so that
$\mathbb{Q}\left[\alpha\right]\subseteq\mathbb{Q}\left[\sqrt{D}\right]$.
Conversely, observe that $\sqrt{D}=\pm\frac{2\alpha+m}{D_0}$ so that
$\mathbb{Q}\left[\sqrt{D}\right]\subseteq\mathbb{Q}\left[\alpha\right]$.

Finally, to see that $R$ is a field it suffices to see that
\[\frac{1}{\alpha}=\frac{2}{\pm D_0\sqrt{D}-m}
=\frac{2\left(\pm D_0\sqrt{D}+m\right)}{m^2-4n-m^2}
=\frac{\mp D_0\sqrt{D}-m}{2n}
\in\mathbb{Q}\left[\sqrt{D}\right]=R.\]

\item %7
Suppose that $D\equiv 2,3\pmod{4}$ and that
$\alpha\in\mathbb{Q}\left[\sqrt{D}\right]$ is an algebraic
integer. Then $\alpha$ satisfies a polynomial
$x^2+mx+n\in\mathbb{Z}\left[x\right]$ and
\begin{equation}\label{Alpha}
\alpha=\frac{-m\pm\sqrt{m^2-4n}}{2}.
\end{equation}
Since $\alpha\in\mathbb{Q}\left[\sqrt{D}\right]$ we know that
$\sqrt{m^2-4n}=a+b\sqrt{D}$ for some $a,b\in\mathbb{Q}$.
Squaring both sides yields
\begin{equation}\label{Square}
m^2-4n=a^2+b^2D+2ab\sqrt{D}.
\end{equation}
Since $m^2-4n\in\mathbb{Z}$
it follows that either $a=0$ or $b=0$.
\begin{enumerate}
\item If $a=0$ then \autoref{Square}
becomes $m^2-4n=b^2D$. Recall that squares
of integers are congruent to either $0$ or $1\pmod{4}$.
If $b^2\equiv 1\pmod{4}$ then $b^2D\equiv 2,3$ so that $m^2\equiv 2,3$,
which is impossible. Thus $b^2\equiv 0$, which implies that
$m^2\equiv 0$. Thus both of $b,m$ are divisible by $2$ so that
$\alpha=\frac{-m\pm b\sqrt{D}}{2}\in\mathbb{Z}\left[\sqrt{D}\right]$.
\item If $b=0$ then \autoref{Square} becomes
$m^2-4n=a^2$. Then $m^2\equiv 0$ if and only if $a^2\equiv 0$.
If $m^2\equiv a^2\equiv 0$ then $m,a$ are both even
and $\alpha=\frac{-m\pm a}{2}
\in\mathbb{Z}\left[\sqrt{D}\right]$.
If $m^2\equiv a^2\equiv 1$ then $m,a$ are both odd
again $\alpha=\frac{-m\pm a}{2}
\in\mathbb{Z}\left[\sqrt{D}\right]$.
\end{enumerate}

\end{enumerate}
\end{document}
