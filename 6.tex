\documentclass[12pt]{article}
\usepackage{multicol,graphicx}
\usepackage[colorlinks,breaklinks,linkcolor=red,citecolor=blue]
{hyperref} 
\def\sectionautorefname~#1\null{\S#1\null}
\usepackage{charter,amsmath,amssymb,breakurl}
\def\equationautorefname~#1\null{(#1)\null}
\def\itemautorefname~#1\null{(#1)\null}
\usepackage{eulervm}
\usepackage[letterpaper,margin=.75in]{geometry}
\renewcommand{\pmod}[1]{\left(\mathsf{mod}\;#1\right)}
\title{Chapter 6}
\renewcommand{\theenumi}{6.\arabic{enumi}}
\author{}\date{}
\let\ord\relax\DeclareMathOperator{\ord}{\mathsf{ord}}
\let\ln\relax\DeclareMathOperator{\ln}{\mathsf{ln}}
\let\deg\relax\DeclareMathOperator{\deg}{\mathsf{deg}}
\let\sin\relax\DeclareMathOperator{\sin}{\mathsf{sin}}
\let\arctan\relax\DeclareMathOperator{\arctan}{\mathsf{arctan}}
\let\cos\relax\DeclareMathOperator{\cos}{\mathsf{cos}}
\let\sec\relax\DeclareMathOperator{\sec}{\mathsf{sec}}
\let\min\relax\DeclareMathOperator*{\min}{\mathsf{min}}
\let\max\relax\DeclareMathOperator*{\max}{\mathsf{max}}
\let\sup\relax\DeclareMathOperator*{\sup}{\mathsf{sup}}
\let\inf\relax\DeclareMathOperator*{\inf}{\mathsf{inf}}
\let\lim\relax\DeclareMathOperator*{\lim}{\mathsf{lim}}
\everymath{\displaystyle}
\begin{document}
\maketitle

\begin{enumerate}
\item %1
Write $\alpha=\sqrt{2}+\sqrt{3}$.
We calculate $\alpha^2=5+2\sqrt{6}$
and $\alpha^4=49+10\sqrt{6}$.
It follows that $\alpha$ is a root of
$x^4-10x^2+1$, so $\alpha$ is an algebraic integer.

\item %2
Suppose
$x^n+a_{n-1}x^{n-1}+\cdots+a_0$ is the
minimal polynomial of $\alpha$.
Evaluating at $x=\alpha$ and multiplying by $d^n$
where $d$ is a common multiple of the denominators
of $a_0,a_1,\ldots,a_{n-1}$ gives
\[0=d^n\alpha^n+a_{n-1}\alpha^{n-1}d^n+\cdots+a_0d^n
=\left(d\alpha\right)^n+a_{n-1}d\left(d\alpha\right)^{n-1}
+\cdots+a_0d^n.\]
This shows that $d\alpha$ is a root of the monic polynomial
$x^n+a_{n-1}dx^{n-1}+\cdots+a_0d^n$
so that $d\alpha$ is an algebraic integer.

\end{enumerate}
\end{document}
