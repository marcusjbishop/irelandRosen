\documentclass[12pt]{article}
\usepackage{multicol,graphicx}
\usepackage[colorlinks,breaklinks,linkcolor=red,citecolor=blue]
{hyperref} 
\def\sectionautorefname~#1\null{\S#1\null}
\usepackage{charter,amsmath,amssymb,breakurl}
\def\equationautorefname~#1\null{(#1)\null}
\def\itemautorefname~#1\null{(#1)\null}
\usepackage{eulervm}
\usepackage[letterpaper,margin=.75in]{geometry}
\renewcommand{\pmod}[1]{\left(\mathsf{mod}\;#1\right)}
\title{Chapter 3}
\author{}\date{}
\let\ln\relax\DeclareMathOperator{\ln}{\mathsf{ln}}
\let\sin\relax\DeclareMathOperator{\sin}{\mathsf{sin}}
\let\arctan\relax\DeclareMathOperator{\arctan}{\mathsf{arctan}}
\let\cos\relax\DeclareMathOperator{\cos}{\mathsf{cos}}
\let\sec\relax\DeclareMathOperator{\sec}{\mathsf{sec}}
\let\min\relax\DeclareMathOperator*{\min}{\mathsf{min}}
\let\max\relax\DeclareMathOperator*{\max}{\mathsf{max}}
\let\sup\relax\DeclareMathOperator*{\sup}{\mathsf{sup}}
\let\inf\relax\DeclareMathOperator*{\inf}{\mathsf{inf}}
\let\lim\relax\DeclareMathOperator*{\lim}{\mathsf{lim}}
\everymath{\displaystyle}
\begin{document}
\maketitle
\thispagestyle{empty}

\begin{enumerate}
\item %1
\item %2
\item %3
\begin{equation}\label{SumDigits}
a+b+c\equiv a+b+c+9b+99c\equiv a+10b+100c\pmod{3}
\end{equation}
so $a+b+c$ is divisible by $3$ if and only $a+10b+100c$
is. \autoref{SumDigits} also holds $\pmod{9}$ so the
same result holds with $3$ replaced with $9$. Similarly
\[a-b+c\equiv a+b+c+11b+99c\equiv a+10b+100c\pmod{11}\]
so $a-b+c$ is divisible by $11$ if and only $a+10b+100c$ is.

\end{enumerate}
\end{document}
