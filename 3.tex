\documentclass[12pt]{article}
\usepackage{multicol,graphicx}
\usepackage[colorlinks,breaklinks,linkcolor=red,citecolor=blue]
{hyperref} 
\def\sectionautorefname~#1\null{\S#1\null}
\usepackage{charter,amsmath,amssymb,breakurl}
\def\equationautorefname~#1\null{(#1)\null}
\def\itemautorefname~#1\null{(#1)\null}
\usepackage{eulervm}
\usepackage[letterpaper,margin=.75in]{geometry}
\renewcommand{\pmod}[1]{\left(\mathsf{mod}\;#1\right)}
\title{Chapter 3}
\author{}\date{}
\let\ln\relax\DeclareMathOperator{\ln}{\mathsf{ln}}
\let\sin\relax\DeclareMathOperator{\sin}{\mathsf{sin}}
\let\arctan\relax\DeclareMathOperator{\arctan}{\mathsf{arctan}}
\let\cos\relax\DeclareMathOperator{\cos}{\mathsf{cos}}
\let\sec\relax\DeclareMathOperator{\sec}{\mathsf{sec}}
\let\min\relax\DeclareMathOperator*{\min}{\mathsf{min}}
\let\max\relax\DeclareMathOperator*{\max}{\mathsf{max}}
\let\sup\relax\DeclareMathOperator*{\sup}{\mathsf{sup}}
\let\inf\relax\DeclareMathOperator*{\inf}{\mathsf{inf}}
\let\lim\relax\DeclareMathOperator*{\lim}{\mathsf{lim}}
\everymath{\displaystyle}
\renewcommand{\theenumi}{3.\arabic{enumi}}
\begin{document}
\maketitle
\thispagestyle{empty}

\begin{enumerate}
\item %1
Inspection of the multiplication table for $\mathbb{Z}/3\mathbb{Z}$
shows that $\overline{5}$ occurs only as the products
$\overline{1}\cdot\overline{5}=\overline{5}\cdot\overline{1}$
and $\overline{1}$ occurs only as the products
$\overline{1}\cdot\overline{1}=\overline{5}\cdot\overline{5}$.
It follows that all the divisors of a number congruent to $5$
must be congruent to either $1$ or $5$. However,
$\overline{1}\cdot\overline{1}=\overline{1}$ shows that
at least one prime factor of a number congruent to $5$
must be congruent to $5$.

Now suppose that only finitely many primes are congruent
to $5$, say $p_1=5,p_2=11,\cdots,p_k$. Then the number
$N=5p_2p_3\cdots p_k+5$ is congruent to $5$.
However, $N$ is not divisible by $p_1=5$ since $p_2\cdots p_k$
is not. Also, $N$ is not divisible by any of $p_2,p_3,\ldots,p_k$
since $5$ is not. This contradicts the assertion above that
$N$ is divisible by a prime congruent to $5$.

\item %2
\item %3
\begin{equation}\label{SumDigits}
a+b+c\equiv a+b+c+9b+99c\equiv a+10b+100c\pmod{3}
\end{equation}
so $a+b+c$ is divisible by $3$ if and only $a+10b+100c$
is. \autoref{SumDigits} also holds $\pmod{9}$ so the
same result holds with $3$ replaced with $9$. Similarly
\[a-b+c\equiv a+b+c+11b+99c\equiv a+10b+100c\pmod{11}\]
so $a-b+c$ is divisible by $11$ if and only $a+10b+100c$ is.

\item\label{Exercise4} %4
If $\left(x,y\right)\in\mathbb{Z}\times\mathbb{Z}$
is a solution to $3x^2+2=y^2$ then $2\equiv y^2\pmod{3}$.
By inspecting the multiplication table
in $\mathbb{Z}/3\mathbb{Z}$ we see
that this is impossible.

\item %5
This is the same as \autoref{Exercise4}.

\item\label{RRS} %6
Note that $a$ is a unit in $\mathbb{Z}/n\mathbb{Z}$
since $\left(a,n\right)=1$. Therefore
$a_i\equiv a_j\pmod{n}$ if and only if $aa_i\equiv aa_j
\pmod{n}$. This shows that
the numbers $aa_1,\ldots,aa_{\phi\left(n\right)}$
are pairwise incongruent $\pmod{n}$. Put
$d=\left(aa_i,n\right)$. Then there exist $q,r\in\mathbb{Z}$
such that
\begin{equation}\label{CongruenceSystem}
d=aa_iq+nr.\end{equation}
But \autoref{CongruenceSystem} also shows that
$d$ divides $\left(a,n\right)=1$ so that $d=1$.

\item %7
First observe that if $a_1,\ldots,a_{\phi\left(n\right)}$
is a reduced residue system, then 
$a_1,\ldots,a_{\phi\left(n\right)}$ represent the distinct
classes of units in $\mathbb{Z}/n\mathbb{Z}$.
It follows that if $b_1,\ldots,b_{\phi\left(n\right)}$
is another reduced residue system, then
$a_1a_2\cdots a_{\phi\left(n\right)}
\equiv b_1b_2\cdots b_{\phi\left(n\right)}\pmod{n}$.
Now taking $b_i=aa_i$ for all $1\le i\le\phi\left(n\right)$
where $a$ is any number such that $\left(a,n\right)=1$,
we see from \autoref{RRS} that
$b_1,\ldots,b_{\phi\left(n\right)}$
is also a reduced residue system. Thus
\[a_1a_2\cdots a_{\phi\left(n\right)}
\equiv b_1b_2\cdots b_{\phi\left(n\right)}\equiv
a^{\phi\left(n\right)}a_1a_2\cdots a_{\phi\left(n\right)}
\pmod{n}.\]
Now since
$a_1a_2\cdots a_{\phi\left(n\right)}$ is a unit, we have
$a^{\phi\left(n\right)}\equiv 1\pmod{n}$.

\item\label{BK} %8
Since $\mathbb{Z}/p\mathbb{Z}$ is a field, all of its
nonzero elements are units. This means that
\begin{equation}\label{PRRS}
1,2,\ldots,p-1
\end{equation}
is a reduced residue system $\pmod{p}$. It follows
that for any $k$ in \autoref{PRRS} there exists
a unique element $b_k$ of \autoref{PRRS} such that
$kb_k\equiv 1\pmod{p}$.
Note that $k=b_k$ if and only if $0\equiv k^2-1\equiv
\left(k+1\right)\left(k-1\right)\pmod{p}$.
This means that $k=b_k$ if and only if $k\equiv 1$
or $k\equiv -1\pmod{p}$.

\item %9
By~\autoref{BK} every factor $k$ of $\left(p-1\right)!$ corresponds
with a unique factor $b_k$ for which $kb_k\equiv 1\pmod{p}$.
Furthermore, $k\ne b_k$ for every $k$ except $k=1$ and $k=p-1$.
It follows that $\left(p-1\right)!\equiv p-1\pmod{p}$.

\item %10
Assume that $n\ne 4$ and
let $d$ be the maximimal proper divisor of $n$ and put $q'=n/d$.
Observe that if $2q'\ge n$ then
$n\le 2q'=\frac{2n}{d}$ so that $d\le 2$. It follows that $n=4$,
which is impossible. Thus $2q'<n$. Take $q$ to be one
of $q'$ or $2q'$ distinct from $d$. Thus $n$ divides
$dq$ which in turn divides $\left(n-1\right)!$ since $d,q$ are distinct
factors of the latter.

\end{enumerate}
\end{document}
