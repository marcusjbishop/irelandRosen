\documentclass[12pt]{article}
\usepackage{multicol,graphicx}
\usepackage[colorlinks,breaklinks,linkcolor=red,citecolor=blue]
{hyperref} 
\def\sectionautorefname~#1\null{\S#1\null}
\usepackage{charter,amsmath,amssymb,breakurl}
\def\equationautorefname~#1\null{(#1)\null}
\def\itemautorefname~#1\null{(#1)\null}
\usepackage{eulervm}
\usepackage[letterpaper,margin=.75in]{geometry}
\renewcommand{\pmod}[1]{\left(\mathsf{mod}\;#1\right)}
\title{Chapter 5}
\renewcommand{\theenumi}{5.\arabic{enumi}}
\author{}\date{}
\let\ord\relax\DeclareMathOperator{\ord}{\mathsf{ord}}
\let\ln\relax\DeclareMathOperator{\ln}{\mathsf{ln}}
\let\deg\relax\DeclareMathOperator{\deg}{\mathsf{deg}}
\let\sin\relax\DeclareMathOperator{\sin}{\mathsf{sin}}
\let\arctan\relax\DeclareMathOperator{\arctan}{\mathsf{arctan}}
\let\cos\relax\DeclareMathOperator{\cos}{\mathsf{cos}}
\let\sec\relax\DeclareMathOperator{\sec}{\mathsf{sec}}
\let\min\relax\DeclareMathOperator*{\min}{\mathsf{min}}
\let\max\relax\DeclareMathOperator*{\max}{\mathsf{max}}
\let\sup\relax\DeclareMathOperator*{\sup}{\mathsf{sup}}
\let\inf\relax\DeclareMathOperator*{\inf}{\mathsf{inf}}
\let\lim\relax\DeclareMathOperator*{\lim}{\mathsf{lim}}
\everymath{\displaystyle}

\makeatletter
\def\legendre@dash#1#2{\hb@xt@#1{%
  \kern-#2\p@
  \cleaders\hbox{\kern.5\p@
    \vrule\@height.2\p@\@depth.2\p@\@width\p@
    \kern.5\p@}\hfil
  \kern-#2\p@
  }}
\def\@legendre#1#2#3#4#5{\mathopen{}\left(
  \sbox\z@{$\genfrac{}{}{0pt}{#1}{#3#4}{#3#5}$}%
  \dimen@=\wd\z@
  \kern-\p@\vcenter{\box0}\kern-\dimen@\vcenter{\legendre@dash\dimen@{#2}}\kern-\p@
  \right)\mathclose{}}
\newcommand\legendre[2]{\mathchoice
  {\@legendre{0}{1}{}{#1}{#2}}
  {\@legendre{1}{.5}{\vphantom{1}}{#1}{#2}}
  {\@legendre{2}{0}{\vphantom{1}}{#1}{#2}}
  {\@legendre{3}{0}{\vphantom{1}}{#1}{#2}}
}
\def\dlegendre{\@legendre{0}{1}{}}
\def\tlegendre{\@legendre{1}{0.5}{\vphantom{1}}}
\makeatother

\begin{document}
\maketitle
\thispagestyle{empty}

\begin{enumerate}
\item %1
$\left(5,10,15\right)\equiv\left(-2,3,1\right)\pmod{7}$
so $\mu=1$ and $\tlegendre{5}{7}=-1$.

$\left(3,6,9,12,15\right)\equiv\left(3,-5,-2,1,4\right)\pmod{11}$
so $\mu=2$ and $\tlegendre{3}{11}=1$.

$\left(6,12,18,24,30,36\right)\equiv\left(6,-1,5,-2,4,-3\right)\pmod{13}$
so $\mu=3$ and $\tlegendre{3}{11}=-1$.

\item %2
If $a$ is a residue then there are $2=1+\dlegendre{a}{p}$ solutions.
If $a$ is a nonresidue then there are $0=1+\dlegendre{a}{p}$ solutions.
If $p\mid a$ then there is $1=1+\dlegendre{a}{p}$ solution,
namely $x=0$.

\item %3
Since $p\nmid a$ we can factor out $a$ and complete the square.
This produces
\begin{equation}\label{QuadraticFormula}
\left(x+\frac{b}{2a}\right)^2\equiv\frac{b^2-4ac}{\left(2a\right)^2.}
\end{equation}
Now if $p\mid b^2-4ac$ then \autoref{QuadraticFormula} has
one solution, namely $x\equiv -\frac{b}{2a}$ and accordingly
$1=1+\dlegendre{b^2-4ac}{p}$. Otherwise assume that
$p\nmid b^2-4ac$.
If $y^2\equiv b^2-4ac$ is solvable, then it has two solutions $y$
leading to the two solutions $x=\frac{y-b}{2a}$
of \autoref{QuadraticFormula}. Accordingly
$2=1+\dlegendre{b^2-4ac}{p}$ in this situation. Finally
$0=1+\dlegendre{b^2-4ac}{p}$ if there are no solutions to
$y^2\equiv b^2-4ac$.

\item\label{SumLegendre} %4
$\dlegendre{a}{p}=1$ if and only if
$x^2\equiv a\pmod{p}$ is solvable if and only if
$a^{\frac{p-1}{2}}\equiv 1\pmod{p}$. The latter has
$\left(p-1,\frac{p-1}{2}\right)=\frac{p-1}{2}$ solutions $a$.
Thus $\frac{p-1}{2}$ of the numbers $1,2,\ldots,p-1$ are
residues $\pmod{p}$ while the remaining $\frac{p-1}{2}$ are nonresidues.
It follows that $\sum_{a=1}^{p-1}\dlegendre{a}{p}=0$.

\item %5
We claim that the numbers $0,a,2a,3a,\ldots,\left(p-1\right)a$
are pairwise inequivalent $\pmod{p}$. Indeed, if $0\le j,k\le p-1$
and $ja\equiv ka\pmod{p}$ then $p\mid a\left(j-k\right)$.
So $p\mid j-k$ since $p\nmid a$. This is only possible if $j=k$
as $\left|j-k\right|<p$.

Similarly
$b,b+a,b+2a,\ldots,b+\left(p-1\right)a$ are pairwise inequivalent
$\pmod{p}$. This means that
\[\left\{b,b+a,b+2a,\ldots,b+\left(p-1\right)a\right\}
=\left\{0,1,2,\ldots,p-1\right\}\] so that
$\sum_{x=0}^{p-1}\dlegendre{ax+b}{p}
=\sum_{j=0}^{p-1}\dlegendre{j}{p}
=\dlegendre{0}{p}+\sum_{j=1}^{p-1}\dlegendre{j}{p}=0$
by \autoref{SumLegendre}.

\end{enumerate}
\end{document}
