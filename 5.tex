\documentclass[12pt]{article}
\usepackage{multicol,graphicx}
\usepackage[colorlinks,breaklinks,linkcolor=red,citecolor=blue]
{hyperref} 
\def\sectionautorefname~#1\null{\S#1\null}
\usepackage{charter,amsmath,amssymb,breakurl}
\def\equationautorefname~#1\null{(#1)\null}
\def\itemautorefname~#1\null{(#1)\null}
\usepackage{eulervm}
\usepackage[letterpaper,margin=.75in]{geometry}
\renewcommand{\pmod}[1]{\left(\mathsf{mod}\;#1\right)}
\title{Chapter 5}
\renewcommand{\theenumi}{5.\arabic{enumi}}
\author{}\date{}
\let\ord\relax\DeclareMathOperator{\ord}{\mathsf{ord}}
\let\ln\relax\DeclareMathOperator{\ln}{\mathsf{ln}}
\let\deg\relax\DeclareMathOperator{\deg}{\mathsf{deg}}
\let\sin\relax\DeclareMathOperator{\sin}{\mathsf{sin}}
\let\arctan\relax\DeclareMathOperator{\arctan}{\mathsf{arctan}}
\let\cos\relax\DeclareMathOperator{\cos}{\mathsf{cos}}
\let\sec\relax\DeclareMathOperator{\sec}{\mathsf{sec}}
\let\min\relax\DeclareMathOperator*{\min}{\mathsf{min}}
\let\max\relax\DeclareMathOperator*{\max}{\mathsf{max}}
\let\sup\relax\DeclareMathOperator*{\sup}{\mathsf{sup}}
\let\inf\relax\DeclareMathOperator*{\inf}{\mathsf{inf}}
\let\lim\relax\DeclareMathOperator*{\lim}{\mathsf{lim}}
\everymath{\displaystyle}

\makeatletter
\def\legendre@dash#1#2{\hb@xt@#1{%
  \kern-#2\p@
  \cleaders\hbox{\kern.5\p@
    \vrule\@height.2\p@\@depth.2\p@\@width\p@
    \kern.5\p@}\hfil
  \kern-#2\p@
  }}
\def\@legendre#1#2#3#4#5{\mathopen{}\left(
  \sbox\z@{$\genfrac{}{}{0pt}{#1}{#3#4}{#3#5}$}%
  \dimen@=\wd\z@
  \kern-\p@\vcenter{\box0}\kern-\dimen@\vcenter{\legendre@dash\dimen@{#2}}\kern-\p@
  \right)\mathclose{}}
\newcommand\legendre[2]{\mathchoice
  {\@legendre{0}{1}{}{#1}{#2}}
  {\@legendre{1}{.5}{\vphantom{1}}{#1}{#2}}
  {\@legendre{2}{0}{\vphantom{1}}{#1}{#2}}
  {\@legendre{3}{0}{\vphantom{1}}{#1}{#2}}
}
\def\dlegendre{\@legendre{0}{1}{}}
\def\tlegendre{\@legendre{1}{0.5}{\vphantom{1}}}
\makeatother

\begin{document}
\maketitle
\thispagestyle{empty}

\begin{enumerate}
\item %1
$\left(5,10,15\right)\equiv\left(-2,3,1\right)\pmod{7}$
so $\mu=1$ and $\tlegendre{5}{7}=-1$.

$\left(3,6,9,12,15\right)\equiv\left(3,-5,-2,1,4\right)\pmod{11}$
so $\mu=2$ and $\tlegendre{3}{11}=1$.

$\left(6,12,18,24,30,36\right)\equiv\left(6,-1,5,-2,4,-3\right)\pmod{13}$
so $\mu=3$ and $\tlegendre{3}{11}=-1$.

To calculate $\tlegendre{-1}{p}$ we observe that $\mu$ is the
number of negative entries in the list
\[-1,-2,-3,-4,-5,\ldots,-\frac{p-1}{2}.\]
If $p=4k+1$ for some $k$ then $\mu=\frac{p-1}{2}=2k$ is even
so $\tlegendre{-1}{p}=1$.
If $p=4k+3$ for some $k$ then $\mu=\frac{p-1}{2}=2k+1$ is odd
so $\tlegendre{-1}{p}=-1$.

\item\label{NumberSolutionsQuadratic} %2
If $a$ is a residue then there are $2=1+\dlegendre{a}{p}$ solutions.
If $a$ is a nonresidue then there are $0=1+\dlegendre{a}{p}$ solutions.
If $p\mid a$ then there is $1=1+\dlegendre{a}{p}$ solution,
namely $x=0$.

\item %3
Since $p\nmid a$ we can factor out $a$ and complete the square.
This produces
\begin{equation}\label{QuadraticFormula}
\left(x+\frac{b}{2a}\right)^2\equiv\frac{b^2-4ac}{\left(2a\right)^2.}
\end{equation}
Now if $p\mid b^2-4ac$ then \autoref{QuadraticFormula} has
one solution, namely $x\equiv -\frac{b}{2a}$ and accordingly
$1=1+\dlegendre{b^2-4ac}{p}$. Otherwise assume that
$p\nmid b^2-4ac$.
If $y^2\equiv b^2-4ac$ is solvable, then it has two solutions $y$
leading to the two solutions $x=\frac{y-b}{2a}$
of \autoref{QuadraticFormula}. Accordingly
$2=1+\dlegendre{b^2-4ac}{p}$ in this situation. Finally
$0=1+\dlegendre{b^2-4ac}{p}$ if there are no solutions to
$y^2\equiv b^2-4ac$.

\item\label{SumLegendre} %4
$\dlegendre{a}{p}=1$ if and only if
$x^2\equiv a\pmod{p}$ is solvable if and only if
$a^{\frac{p-1}{2}}\equiv 1\pmod{p}$. The latter has
$\left(p-1,\frac{p-1}{2}\right)=\frac{p-1}{2}$ solutions $a$.
Thus $\frac{p-1}{2}$ of the numbers $1,2,\ldots,p-1$ are
residues $\pmod{p}$ while the remaining $\frac{p-1}{2}$ are nonresidues.
It follows that $\sum_{a=1}^{p-1}\dlegendre{a}{p}=0$.

\item %5
We claim that the numbers $0,a,2a,3a,\ldots,\left(p-1\right)a$
are pairwise inequivalent $\pmod{p}$. Indeed, if $0\le j,k\le p-1$
and $ja\equiv ka\pmod{p}$ then $p\mid a\left(j-k\right)$.
So $p\mid j-k$ since $p\nmid a$. This is only possible if $j=k$
as $\left|j-k\right|<p$.

Similarly
$b,b+a,b+2a,\ldots,b+\left(p-1\right)a$ are pairwise inequivalent
$\pmod{p}$. This means that
\[\left\{b,b+a,b+2a,\ldots,b+\left(p-1\right)a\right\}
=\left\{0,1,2,\ldots,p-1\right\}\] so that
$\sum_{x=0}^{p-1}\dlegendre{ax+b}{p}
=\sum_{j=0}^{p-1}\dlegendre{j}{p}
=\dlegendre{0}{p}+\sum_{j=1}^{p-1}\dlegendre{j}{p}=0$
by \autoref{SumLegendre}.

\item %6
For each $0\le y\le p-1$ 
the congruence $x^2-y^2\equiv a\pmod{p}$ 
has $1+\dlegendre{y^2+a}{p}$ solutions
by \autoref{NumberSolutionsQuadratic}.
It follows that $x^2-y^2\equiv a$ has
$\sum_{y=0}^{p-1}\left(1+\dlegendre{y^2+a}{p}\right)$ solutions.

\item %7
Under the change of variables $\begin{cases}u&=x+y\\v&=x-y\end{cases}$
the congruence $x^2-y^2\equiv a\pmod{p}$ becomes
\begin{equation}\label{VariableChange}
uv\equiv a\pmod{p}.
\end{equation}
If $p\mid a$ then either $u\equiv 0$ or $v\equiv 0$ (or both)
leading to $2p-1$ solutions to
\autoref{VariableChange}.
If $p\nmid a$ then \autoref{VariableChange} has $p-1$ solutions,
namely $\left(u,v\right)=\left(z,\frac{a}{z}\right)$
with $1\le z\le p-1$.

\item %8
\begin{align*}
\sum_{y=0}^{p-1}\dlegendre{y^2+a}{p}
&=-p+\sum_{y=0}^{p-1}\left(1+\dlegendre{y^2+a}{p}\right)\\
&=-p+\begin{cases}2p-1&p\mid a\\p-1&p\nmid a\end{cases}\\
&=\begin{cases}p-1&p\mid a\\-1&p\nmid a\end{cases}
\end{align*}

\item\label{ProductOddSquares} %9
\begin{align*}
1^23^2\cdots\left(p-1\right)^2
&\equiv\left(-1\right)^{\frac{p-2}{2}}
1\left(-1\right)3\left(-3\right)
\cdots\left(p-2\right)\left(-\left(p-2\right)\right)\\
&\equiv\left(-1\right)^{\frac{p-1}{2}}
1\left(p-1\right)3\left(p-3\right)
\cdots\left(p-2\right)\left(2\right)\\
&\equiv\left(-1\right)^{\frac{p-1}{2}}
\left(p-1\right)!\\
&\equiv\left(-1\right)^{\frac{p+1}{2}}
\end{align*}
As a remark, recall
that there are exactly $\frac{p-1}{2}$
residues, but for each $1\le k\le p-1$ both of
$k,p-k$ have the same square. Exactly one of $k,p-k$ being even and
the other odd, if follows that the residues are the squares of the 
even elements of $1,2,\ldots,p-1$. The residues are also the squares
of the odd elements of $1,2,\ldots,p-1$. This means that the
expression $1^23^2\cdots\left(p-1\right)^2$ in this exercise
is simply the product of the residues.

\item %10
By \autoref{ProductOddSquares}
\[r_1r_2\cdots r_{\frac{p-1}{2}}=\left(-1\right)^{\frac{p+1}{2}}
=\begin{cases}1&\text{if $p\equiv 3\pmod{4}$}\\
-1&\text{if $p\equiv 1\pmod{4}$}\end{cases}.\]
Another way to do this is to use a primitive root $g\pmod{p}$.
The residues are simply the even powers of $g$, so
\[r_1r_2\cdots r_{\frac{p-1}{2}}
=\prod_{k=1}^{\frac{p-1}{2}}g^{2k}
=g^{\sum_{k=1}^{\frac{p-1}{2}}2k}
=g^{\frac{p-1}{2}\frac{p+1}{2}}
=\left(-1\right)^\frac{p+1}{2}.\]

\item %11
Since $p\equiv 3\pmod{4}$ it follows that
$q=2p+1\equiv 7\pmod{8}$. This means that
$2=b^2\pmod{q}$ for some $b\in\mathbb{Z}$
by Quadratic Reciprocity. Then
\begin{equation}\label{FermatFactorization}
2^p-1\equiv b^{2p}-1=\left(b^p-1\right)\left(b^p+1\right)
=\left(b^{\frac{q-1}{2}}-1\right)\left(b^{\frac{q-1}{2}}+1\right)
\pmod{q}.
\end{equation}
But $b^{q-1}\equiv 1\pmod{q}$ by Fermat's Little Theorem
so that $b^{\frac{q-1}{2}}\equiv\pm 1\pmod{q}$.
Then \autoref{FermatFactorization} shows that $q$ divides $2^p-1$,
so $2^p-1$ is not prime.

\item %12
Naturally $2$ divides $x^2+1$ since $2\mid\left(n^2+1\right)$
for any odd $n$. Suppose otherwise that $p$ is an odd prime. Then
$p\mid\left(n^2+1\right)$ for some $n$ if and only if
$n^2\equiv -1\pmod{p}$ for some $n$ if and only if
$\dlegendre{-1}{p}=1$ if and only if $p\equiv 1\pmod{4}$.

Naturally $2$ divides $x^2-2$ since $2\mid\left(n^2-2\right)$
for any even $n$. Suppose otherwise that $p$ is an odd prime. Then
$p\mid\left(n^2-2\right)$ for some $n$ if and only if
$n^2\equiv 2\pmod{p}$ for some $n$ if and only if
$\dlegendre{2}{p}=1$ if and only if $p\equiv 1,7\pmod{8}$.

\item %13
\item %14
Assuming $p\equiv 1\pmod{3}$ the order of
$U\left(\mathbb{Z}/p\mathbb{Z}\right)$ is divisible by $3$.
This group being cyclic, it has an element $\rho$ of order $3$.
Then $\left(2\rho+1\right)^2=4\rho^2+4\rho+1
=4\left(\rho^2+\rho+1\right)-3\equiv -3\pmod{p}$
since $\rho^2+\rho+1=\frac{1-\rho^3}{1-\rho}\equiv 0$.

\item %15
Assuming $p\equiv 1\pmod{5}$ the order of
$U\left(\mathbb{Z}/p\mathbb{Z}\right)$ is divisible by $5$.
This group being cyclic, it has an element $\rho$ of order $5$.
Putting $y=\rho^4+\rho$ we see that
\begin{equation}\label{RhoDependence}
y^2+y-1=\rho^8+2\rho^5+\rho^2+\rho^4+\rho-1
=\rho^4+\rho^3+\rho^2+\rho+1\equiv 0\pmod{p}.
\end{equation}
This means that
\[\left(2y+1\right)^2=4y^2+4y+1=4\left(y^2+y\right)+1
\equiv 4\left(1\right)+1=5\pmod{p}\]
so that $\dlegendre{5}{p}=1$.
We remark that we stumbled upon the polynomial $2y+1$
as follows. On the one hand \autoref{RhoDependence}
shows that $\rho^4+\rho$ is a zero
of $y^2+y-1$. But on the other hand
$y=\frac{-1\pm\sqrt{5}}{2}$ by the quadratic formula.
This means that $5$ is a square, and moreover
$\sqrt{5}=\pm\left(2y+1\right)$.

\item %16
$\tlegendre{7}{p}=1$ if and only if $p\equiv\pm b^2\pmod{28}$
for an odd integer $b$ prime to $7$. Since
\[\left\{1^2,3^2,5^2,9^2,11^2,13^2\right\}\equiv\left\{1,9,25\right\}
\pmod{28}\]
we conclude that $7$ is a residue $\pmod{p}$ when
$p\equiv 1,3,9,19,25,27\pmod{28}$.

Next we find the primes $p$ for which $\dlegendre{15}{p}=1$.
Suppose $\dlegendre{3}{p}=1=\dlegendre{5}{p}$.
This means that $p\equiv 1,12\pmod{12}$ and $p\equiv 1,4\pmod{5}$.
\begin{itemize}
\item $p=12k+1=5l+1$ can be solved for $\left(k,l\right)$
whenever $k$ is a multiple of $5$.
This leads to the solution $p\equiv 1\pmod{60}$.
\item $p=12k+1=5l+4$ if and only if $12k=5l+3$. The latter can be solved
whenever $5l+3$ is a multiple of $12$, or $5l\equiv 9\pmod{12}$.
$l\equiv 9\pmod{12}$ is the only solution, leading to
$p=5\left(12j+9\right)+4=60j+49$. Thus $p\equiv 49\pmod{60}$.
\item $p=12k+11=5l+1$ if and only if $12k+10=5l$. The latter can be solved
whenever $12k+10$ is a multiple of $5$.
$k\equiv 0\pmod{5}$ is the only solution, leading to
$p=12\left(5j\right)+11=60j+11$. Thus $p\equiv 11\pmod{60}$.
\item $p=12k+11=5l+4$ if and only if $12k+7=5l$. The latter can be solved
whenever $12k+7$ is a multiple of $5$, or $12k\equiv 3\pmod{5}$.
$k\equiv 4\pmod{5}$ is the only solution, leading to
$p=12\left(5j+4\right)+11=60j+59$. Thus $p\equiv 59\pmod{60}$.
\end{itemize}
Suppose now that $\dlegendre{3}{p}=-1=\dlegendre{5}{p}$.
This means that $p\equiv 5,7\pmod{12}$ and $p\equiv 2,3\pmod{5}$.
\begin{itemize}
\item $12k+5=5l+2$ can be solved if and only if $12k\equiv 2\pmod{5}$.
The solution $k\equiv 1\pmod{5}$ leads to $p=17\pmod{60}$.
\item $12k+5=5l+3$ can be solved if and only if $12k\equiv 3\pmod{5}$.
The solution $k\equiv 4\pmod{5}$ leads to $p=53\pmod{60}$.
\item $12k+7=5l+2$ can be solved if and only if $12k\equiv 0\pmod{5}$.
The solution $k\equiv 0\pmod{5}$ leads to $p=7\pmod{60}$.
\item $12k+7=5l+3$ can be solved if and only if $12k\equiv 1\pmod{5}$.
The solution $k\equiv 3\pmod{5}$ leads to $p=43\pmod{60}$.
\end{itemize}
In summary, we have $\dlegendre{15}{p}$ if $p\equiv
1,7,11,17,43,49,53,59\pmod{60}$.

\item %17
Suppose $b=p_1p_2\cdots p_k$. If $a_1\equiv a_2\pmod{b}$
then $a_1\equiv a_2\pmod{p_i}$ for all $1\le i\le k$ and
\[\dlegendre{a_1}{b}
=\dlegendre{a_1}{p_1}\cdots\dlegendre{a_1}{p_k}
=\dlegendre{a_2}{p_1}\cdots\dlegendre{a_2}{p_k}
=\dlegendre{a_2}{b}.\]
Now for any $a_1,a_2$
\[\dlegendre{a_1a_2}{b}
=\dlegendre{a_1a_2}{p_1}\cdots\dlegendre{a_1a_2}{p_k}
=\dlegendre{a_1}{p_1}\cdots\dlegendre{a_1}{p_k}
\dlegendre{a_2}{p_1}\cdots\dlegendre{a_2}{p_k}
=\dlegendre{a_1}{b}\dlegendre{a_2}{b}.\]
Finally, if
$b_1=p_1p_2\cdots p_k$ and
$b_2=q_1q_2\cdots q_l$ then
\[\dlegendre{a}{b_1b_2}
=\dlegendre{a}{p_1}\cdots\dlegendre{a}{p_k}
\dlegendre{a}{q_1}\cdots\dlegendre{a}{q_l}
=\dlegendre{a}{b_1}\dlegendre{a}{b_2}.\]

Suppose $r_1,r_2,\ldots,r_k$ are odd integers. Then
\begin{align*}
\frac{r_1r_2\cdots r_k-1}{2}
&\equiv\frac{r_1r_2\cdots r_{k-1}-1}{2}+\frac{r_k-1}{2}\\
&\equiv\sum_{j=1}^{k-1}\frac{r_j-1}{2}+\frac{r_k-1}{2}\\
&\equiv\sum_{j=1}^k\frac{r_j-1}{2}\pmod{2}
\end{align*}
by induction. Similarly
\begin{align*}
\frac{r_1^2r_2^2\cdots r_k^2-1}{8}
&\equiv\frac{r_1^2r_2^2\cdots r_{k-1}^2-1}{8}+\frac{r_k^2-1}{8}\\
&\equiv\sum_{j=1}^{k-1}\frac{r_j^2-1}{8}+\frac{r_k^2-1}{8}\\
&\equiv\sum_{j=1}^k\frac{r_j^2-1}{8}\pmod{2}.
\end{align*}

\item %18
The prime factorization of $D$ is $p_1p_2\cdots p_k$
where $p_1,p_2,\ldots,p_k$ are distinct, odd primes. 
Modulo any prime half the units are residues and the other
half are nonresidues, so we can choose integers $a_1,a_2,\ldots,a_k$
with $\dlegendre{a_i}{p_i}=1$ for all $1\le i\le k-1$ and
$\dlegendre{a_k}{p_k}=-1$. By the Chineese Remainder Theorem
the system 
\[\left\{b\equiv a_i\pmod{p_i}\left|1\le i\le k\rule{0pt}{11pt}\right.\right\}\]
has a solution $b$. Then
\[\dlegendre{b}{D}=\dlegendre{b}{p_1}\cdots\dlegendre{b}{p_k}
=\dlegendre{a_1}{p_1}\cdots\dlegendre{a_k}{p_k}=-1.\]
Note that $0\not\equiv\pm 1\equiv a_i\equiv b\pmod{p_i}$ for all $1\le i\le k$
so that $\left(b,D\right)=1$.

\item %19
\item %20
\item %21

\item %22
Since $113\equiv 1\pmod{4}$ we have
\[\dlegendre{113}{997}=\dlegendre{997}{113}
=\dlegendre{113\left(8\right)+93}{113}=\dlegendre{93}{113}
=\dlegendre{3}{113}\dlegendre{31}{113}.\]
Now $\dlegendre{3}{113}=\dlegendre{113}{3}
=\dlegendre{2}{3}=-1$ since $3\equiv 3\pmod{8}$.
Next
\[\dlegendre{31}{113}=\dlegendre{113}{31}
=\dlegendre{20}{31}=\left.\dlegendre{2}{31}\right.^2\dlegendre{5}{31}
=\dlegendre{5}{31}=\dlegendre{31}{5}=\dlegendre{1}{5}=1.\]
We conclude that $\dlegendre{113}{997}=-1$.

Next we compute $\dlegendre{215}{761}=\dlegendre{5}{761}\dlegendre{43}{761}$.
We calculate $\dlegendre{5}{761}=\dlegendre{761}{5}=\dlegendre{1}{5}=1$ and
\begin{multline*}
\dlegendre{43}{761}=\dlegendre{761}{43}
=\dlegendre{30}{43}=\dlegendre{2}{43}\dlegendre{3}{43}\dlegendre{5}{43}
=-\dlegendre{3}{43}\dlegendre{5}{43}\\
=\dlegendre{43}{3}\dlegendre{43}{5}
=\dlegendre{1}{3}\dlegendre{3}{5}=\dlegendre{5}{3}=\dlegendre{2}{3}=-1
\end{multline*}
so that $\dlegendre{215}{761}=-1$.

Two exercises remain\dots

\end{enumerate}
\end{document}
