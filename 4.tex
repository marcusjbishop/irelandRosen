\documentclass[12pt]{article}
\usepackage{multicol,graphicx}
\usepackage[colorlinks,breaklinks,linkcolor=red,citecolor=blue]
{hyperref} 
\def\sectionautorefname~#1\null{\S#1\null}
\usepackage{charter,amsmath,amssymb,breakurl}
\def\equationautorefname~#1\null{(#1)\null}
\def\itemautorefname~#1\null{(#1)\null}
\usepackage{eulervm}
\usepackage[letterpaper,margin=.75in]{geometry}
\renewcommand{\pmod}[1]{\left(\mathsf{mod}\;#1\right)}
\title{Chapter 4}
\renewcommand{\theenumi}{4.\arabic{enumi}}
\author{}\date{}
\let\ord\relax\DeclareMathOperator{\ord}{\mathsf{ord}}
\let\ln\relax\DeclareMathOperator{\ln}{\mathsf{ln}}
\let\deg\relax\DeclareMathOperator{\deg}{\mathsf{deg}}
\let\sin\relax\DeclareMathOperator{\sin}{\mathsf{sin}}
\let\arctan\relax\DeclareMathOperator{\arctan}{\mathsf{arctan}}
\let\cos\relax\DeclareMathOperator{\cos}{\mathsf{cos}}
\let\sec\relax\DeclareMathOperator{\sec}{\mathsf{sec}}
\let\min\relax\DeclareMathOperator*{\min}{\mathsf{min}}
\let\max\relax\DeclareMathOperator*{\max}{\mathsf{max}}
\let\sup\relax\DeclareMathOperator*{\sup}{\mathsf{sup}}
\let\inf\relax\DeclareMathOperator*{\inf}{\mathsf{inf}}
\let\lim\relax\DeclareMathOperator*{\lim}{\mathsf{lim}}
\everymath{\displaystyle}
\begin{document}
\maketitle
\thispagestyle{empty}

\begin{enumerate}
\item %1
The powers of $2\pmod{29}$ are
2, 4, 8, 16, 3, 6, 12, 24, 19, 9,
7, 14, 28, 27, 25, 21, 13, 26, 23, 17, 5, 10, 20, 11, 22, 15, 1.
Since there are $28=\left|U\left(\mathbb{Z}/29\mathbb{Z}\right)\right|$
of these, $2$ is a primitive root $\pmod{29}$.

\item %2
2,6,7,8 are primitive roots $\pmod{11}$.
2,6,7,11 are primitive roots $\pmod{13}$.
3, 5, 6, 7, 10, 11, 12, 14 are primitive roots $\pmod{17}$.
Finally, 2, 3, 10, 13, 14, 15 are primitive roots $\pmod{19}$.

\item %3
Since $a$ is primitive $\pmod{p^n}$ the numbers
\begin{equation}\label{ListPN}
a,a^2,a^3,\ldots,a^{p^{n-1}\left(p-1\right)}
\end{equation}
are distinct $\pmod{p^n}$, only the last element being
congruent to $1\pmod{p^n}$.
Now suppose $j-i<p-1$. Then
\begin{equation}\label{ResiduesPN}
a^i-a^j=a^i\left(1-a^{j-i}\right)
\end{equation}
But $a^i\not\equiv 0\pmod{p}$
since otherwise $a^i\equiv 0\pmod{p^n}$, which is impossible
since $a$ is a primitive root $\pmod{p^n}$.
Furthermore $1-a^{j-i}\not\equiv 0\pmod{p}$ since otherwise
$a^{j-i}\equiv 1\pmod{p^n}$, contradicting the assertion
above that
$a^{p^{n-1}\left(p-1\right)}$ is the only element of
\autoref{ListPN} congruent to $1\pmod{p^n}$.
It follows that $a^i\not\equiv a^j\pmod{p}$ for all
$\left|i-j\right|<p-1$. In particular, $a,a^2,a^3,\ldots,a^{p-1}$
are inequivalent $\pmod{p}$ so that $a$ is a primitive root
$\pmod{p}$.

However, we note that if $j-i=p-1$ then \autoref{ResiduesPN}
shows that $a^i-a^j\equiv 0\pmod{p}$ by Fermat's Little Theorem.
This shows that the elements of \autoref{ListPN}, when reduced
$\pmod{p}$, cycle through the nonzero residues.

\item\label{2T+1} %4
First we note that by Wilson's Theorem
\[-1\equiv\prod_{j=1}^{4t}a^j
\equiv a^{\sum_{j=1}^kj}
\equiv a^{\frac{4t\left(4t+1\right)}{2}}
\equiv a^{2t}\pmod{p}\]
where the last reduction follows from Fermat's Little Theorem.
Now to verify that $-a$ is a primitive root $\pmod{p}$
we need to check that $1\not\equiv\left(-a\right)^k\pmod{p}$
for any $k<4t$. For even $k$ this holds because $a$ is a primitive root.
If $k$ is odd then
$\left(-a\right)^k=-a^k\not\equiv 1$ unless
$a^k\equiv -1$, which is impossible by the observation above
since $2t\ne k$ as $k$ is odd.

\item %5
Suppose $a$ is a primitive root $\pmod{p}$. As in \autoref{2T+1}
we use Wilson's Theorem to calculate $-a\equiv a^{2t+2}\pmod{p}$.
Now if $k$ is such that
\[1\equiv\left(-a\right)^k=a^{k\left(2t+2\right)}\]
then knowing that $4t+2=p-1$ is the smallest exponent $r$ for
which $a^r\equiv 1\pmod{p}$ we conclude that
$4t+2\mid k\left(2t+2\right)$ or $2t+1\mid k\left(t+1\right)$.
Since $2t+1$ and $t+1$ are relatively prime we conclude $2t+1\mid k$.
Thus $2t+1=\frac{p-1}{2}$ is the order of $-a$.

If conversely $a$ has order $2t+1$, then we first observe that
$2k+1$ is not the order of $-a$ since $\left(-a\right)^{2t+1}
\equiv -a^{2t+1}\equiv -1$.
However $\left(-a\right)^{4t+2}=a^{2\left(2t+1\right)}\equiv 1$
so the order of $-a$ divides $4t+2$. Now since no
$k$ with $1\le k\le 4t+2$ divides $4t+2$ other than $2t+1$ or $2$,
we conclude that $4t+2$ is the order of $-a$.

\item %6
\item %7

\item %8
If $a$ is a primitive root $\pmod{p}$ then $a^d\not\equiv 1\pmod{p}$
for {\em all proper} divisors $d$ of $p-1$. In particular $a^d\not\equiv 1$
for divisors of the form $d=\frac{p-1}{q}$ where $q$ is a prime divisor
of $p-1$.

Conversely, suppose that $a^{\frac{p-1}{q}}\not\equiv 1$
for all prime divisors
$q$ of $p-1$. Now if $d$ is any {\em proper} divisor of $p-1$ then
$\ord_q{d}<\ord_q\left(p-1\right)$ for some prime divisor $q$ of $p-1$.
It follows that $d$ divides $\frac{p-1}{q}$. This means that $a^d$ can't be
congruent to $1$, because then $a^{\frac{p-1}{q}}\equiv 1$.
We conclude that $a^d\not\equiv 1$ for all proper divisors $d$ of
$p-1$. This means that $a$ is a primitive root $\pmod{p}$.

\item %9
If $p=2$ then $1$ is the only primitive root $\pmod{2}$
and $1\equiv -1\equiv\left(-1\right)^{\varphi\left(2-1\right)}
\pmod{2}$.
If $p=3$ then $2$ is the only primitive root $\pmod{3}$
and $2\equiv -1\equiv\left(-1\right)^{\varphi\left(3-1\right)}$.
We assume otherwise that $p>3$.

If $r$ is any primitive root $\pmod{p}$ then the remaining
primitive roots are the other generators of 
of $U\left(\mathbb{Z}_p\right)$,
which are $r^k$ with $\left(p-1,k\right)=1$ by \autoref{CyclicGenerators}.
Knowing that $\frac{\left(p-1\right)\varphi\left(p-1\right)}{2}$
is the sum of the least residues $\pmod{p-1}$
prime to $p-1$, it follows that
\[\prod_{\left(p-1,k\right)=1}r^k
=r^{\frac{\left(p-1\right)\varphi\left(k-1\right)}{2}}
=\left(r^{p-1}\right)^{\frac{\varphi\left(p-1\right)}{2}}=1.\]
Now since $\varphi\left(p-1\right)$ is even, it follows that
$1=\left(-1\right)^{\varphi\left(p-1\right)}$ is the product
of the primitive roots $\pmod{p}$.

\item %10
\item %11
\item Let $g$ be a primitive root $\pmod{p}$.
The cyclic group $U\left(\mathbb{Z}/p\mathbb{Z}\right)$
has exactly one element of every order $d$ for every divisor
$d$ of $p-1$. If $d=2$ then this element is $g^{\frac{p-1}{2}}$
since $\left(g^{\frac{p-1}{2}}\right)^2=g^{p-1}\equiv 1\pmod{p}$.
However, $-1$ clearly represents a class of order $2$.
This means that $g^{\frac{p-1}{2}}\equiv -1$.
Finally,
\[\left(p-1\right)!=\prod_{j=1}^{p-1}g^j
=g^{\frac{\left(p-1\right)p}{2}}
=\left(g^p\right)^{\frac{p-1}{2}}
\equiv g^{\frac{p-1}{2}}
\equiv -1\pmod{p}.\]

\item\label{CyclicGenerators} %13
Suppose $\left(n,k\right)=1$.
Consider the subgroup 
$\left\{g^{jk}\mid j\ge 1\right\}$
of $G$ generated by $g^k$.
Since $n$ is the order of $g$ we have $g^{jk}=1$
if and only if $n\mid jk$. As $\left[n,k\right]=nk$
it follows that $g^{nk}=1$ but $g^{jk}\ne 1$ for any $j<n$.
This shows that $n$ is the order of $g^k$
so that $g^k$ is a generator of $G$.

\item %14
Let $k\in\mathbb{Z}$ and write $k=mq_1+r_1=nq_2+r_2$ for
some $q_1,q_2,r_1,r_2\in\mathbb{Z}$ with $0\le r_1<m$ and $0\le r_2<n$. Then 
\[1=\left(ab\right)^k=a^{mq_1+r_1}b^{nq_2+r_2}=a^{r_1}b^{r_2}\]
implies that $a^{r_1}$ is the inverse of $b^{r_2}$.
Now since $a^{r_1}$ has order dividing $m$ and $b^{r_2}$
has order dividing $n$ but $\left(m,n\right)=1$ we must
have $r_1=r_2=0$. This shows that $m$ and $n$ both divide $k$
so that $mn=\left[m,n\right]$ also divides $k$. It follows
that $mn$ is the order of $ab$.

\item %15

\item %16
2 is a primitive root $\pmod{19}$. Then $x^3\equiv 1\pmod{19}$
is equivalent to $2^{3y}\equiv 2^{18}\pmod{19}$
which in turn is equivalent to $3y\equiv 18\pmod{18}$.
$y=6$ is clearly a solution to the latter, and the remaining
$3=\left(3,18\right)$ solutions differ by $\frac{18}{3}=6$.
Thus the solutions to $3y\equiv 18\pmod{18}$ are $y=6,12,18$.
Now since $x=2^y$ the solutions to $x^3\equiv 1\pmod{19}$
are $x=2^6\equiv 7,
2^{12}\equiv 11,2^{18}\equiv 1$. 

Using the same method we find that $x=1,4,13,16$ are the solutions
to $x^4\equiv 1\pmod{17}$.

\item\label{X7E29} %17
If $x=2^y$ then $x^7\equiv 1\pmod{29}$ if and only
if $2^{7y}\equiv 2^{28}\pmod{29}$ which in turn holds
if and only if $7y\equiv 28\pmod{28}$. The latter has solutions
$y=4,8,12,16,20,24,28$ which correspond with solutions
$x=1,7,16,20,23,24,25$ of the original congurence.

\item %18
By \autoref{X7E29}
$x=1,7,16,20,23,24,25$ are the solutions to $x^7-1\equiv 0\pmod{29}$.
Now since $x^7-1=\left(x-1\right)\left(1+x+x^2+x^3+x^4+x^5+x^6\right)$
and $x=1$ is a solution to $x-1\equiv 0$ it follows that
$x=7,16,20,23,24,25$ are the solutions to 
$1+x+x^2+x^3+x^4+x^5+x^6\equiv 0$.


\end{enumerate}
\end{document}
