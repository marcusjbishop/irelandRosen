\documentclass[12pt]{article}
\usepackage{multicol,graphicx}
\usepackage[colorlinks,breaklinks,linkcolor=red,citecolor=blue]
{hyperref} 
\def\sectionautorefname~#1\null{\S#1\null}
\usepackage{charter,amsmath,amssymb,breakurl}
\def\equationautorefname~#1\null{(#1)\null}
\def\itemautorefname~#1\null{(#1)\null}
\usepackage{eulervm}
\usepackage[letterpaper,margin=.75in]{geometry}
\renewcommand{\pmod}[1]{\left(\mathsf{mod}\;#1\right)}
\title{Chapter 4}
\author{}\date{}
\let\ord\relax\DeclareMathOperator{\ord}{\mathsf{ord}}
\let\ln\relax\DeclareMathOperator{\ln}{\mathsf{ln}}
\let\deg\relax\DeclareMathOperator{\deg}{\mathsf{deg}}
\let\sin\relax\DeclareMathOperator{\sin}{\mathsf{sin}}
\let\arctan\relax\DeclareMathOperator{\arctan}{\mathsf{arctan}}
\let\cos\relax\DeclareMathOperator{\cos}{\mathsf{cos}}
\let\sec\relax\DeclareMathOperator{\sec}{\mathsf{sec}}
\let\min\relax\DeclareMathOperator*{\min}{\mathsf{min}}
\let\max\relax\DeclareMathOperator*{\max}{\mathsf{max}}
\let\sup\relax\DeclareMathOperator*{\sup}{\mathsf{sup}}
\let\inf\relax\DeclareMathOperator*{\inf}{\mathsf{inf}}
\let\lim\relax\DeclareMathOperator*{\lim}{\mathsf{lim}}
\everymath{\displaystyle}
\renewcommand{\theenumi}{2.\arabic{enumi}}
\begin{document}
\maketitle
\thispagestyle{empty}

\begin{enumerate}
\item %1
The powers of $2\pmod{29}$ are
2, 4, 8, 16, 3, 6, 12, 24, 19, 9,
7, 14, 28, 27, 25, 21, 13, 26, 23, 17, 5, 10, 20, 11, 22, 15, 1.
Since there are $28=\left|U\left(\mathbb{Z}/29\mathbb{Z}\right)\right|$
of these, $2$ is a primitive root $\pmod{29}$.

\item %2
2,6,7,8 are primitive roots $\pmod{11}$.
2,6,7,11 are primitive roots $\pmod{13}$.
3, 5, 6, 7, 10, 11, 12, 14 are primitive roots $\pmod{17}$.
Finally, 2, 3, 10, 13, 14, 15 are primitive roots $\pmod{19}$.

\item %3
Since $a$ is primitive $\pmod{p^n}$ the numbers
\begin{equation}\label{ListPN}
a,a^2,a^3,\ldots,a^{p^{n-1}\left(p-1\right)}
\end{equation}
are distinct $\pmod{p^n}$, only the last element being
congruent to $1\pmod{p^n}$.
Now suppose $j-i<p-1$. Then
\begin{equation}\label{ResiduesPN}
a^i-a^j=a^i\left(1-a^{j-i}\right)
\end{equation}
But $a^i\not\equiv 0\pmod{p}$
since otherwise $a^i\equiv 0\pmod{p^n}$, which is impossible
since $a$ is a primitive root $\pmod{p^n}$.
Furthermore $1-a^{j-i}\not\equiv 0\pmod{p}$ since otherwise
$a^{j-i}\equiv 1\pmod{p^n}$, contradicting the assertion
above that
$a^{p^{n-1}\left(p-1\right)}$ is the only element of
\autoref{ListPN} congruent to $1\pmod{p^n}$.
It follows that $a^i\not\equiv a^j\pmod{p}$ for all
$\left|i-j\right|<p-1$. In particular, $a,a^2,a^3,\ldots,a^{p-1}$
are inequivalent $\pmod{p}$ so that $a$ is a primitive root
$\pmod{p}$.

However, we note that if $j-i=p-1$ then \autoref{ResiduesPN}
shows that $a^i-a^j\equiv 0\pmod{p}$ by Fermat's Little Theorem.
This shows that the elements of \autoref{ListPN}, when reduced
$\pmod{p}$, cycle through the nonzero residues.
\end{enumerate}
\end{document}
